%-*- coding: UTF-8 -*-
% notes.tex
%
\documentclass[UTF8]{article}
\usepackage{geometry}
\geometry{a4paper, centering, scale=0.8}
\usepackage{minted}
\usepackage{hyperref}
\usepackage{indentfirst} % to indent the first paragraph of a section

\title{The Python Tutorial Notes}
\author{Du Ang \\ \texttt{du2ang233@gmail.com} }
\date{\today}

\begin{document}
\maketitle

\tableofcontents
\newpage

\section{Whetting Your Appetite}
\subsection{Some applications of Python}
\begin{itemize}
    \item Automate works on computer
    \item Help software engineers develop and test more quickly
    \item ...
\end{itemize}

\subsection{The features of Python}
\begin{itemize}
    \item High-level language \\
    High-level more general data types
    \item Allow to split program into modules
    \item Interpreted language
    \begin{itemize}
        \item No necessary compilation and linking
        \item The interpreter can be used interactively
    \end{itemize}
    \item Shorter programs than equivalent C, C++ or Java programs
    \begin{itemize}
        \item high-level data types
        \item statement grouping is done by indentation instead of brackets
        \item no variable or argument declarations are necessary
    \end{itemize}
    \item Extensible
\end{itemize}

\subsection{The origin of the name "Python"}
“Monty Python’s Flying Circus”, the skits on BBC.

\section{Using the Python Interpreter}
\subsection{Invoking the Interpreter}
The Python interpreter is usually installed as \texttt{/usr/local/bin/python}, and usually the
\texttt{/usr/local/bin} has been put in the Unix shell’s search path. Type command \texttt{python}
to the shell to invoke the Python interpreter.

Type an end-of-file character(\texttt{Ctrl-D} on Unix, \texttt{Ctrl-Z} on Windows) to exit the
interpreter. If that doesn’t work, type command \mintinline{Python}{quit()} to the shell.

A second way of starting the interpreter is \mintinline{bash}{python -c 'command [arg] ...'},
which executes the statement(s) in command.

Some Python modules are also useful as scripts. These can be invoked using
\mintinline{bash}{python -m module [arg] ...}, which executes the source file for modules as if you
had spelled out its full name on the command line.

\subsubsection{Argument Passing}
When known to the interpreter, the script name and additional arguments thereafter are turned into
a list of strings and assigned to the \texttt{argv} variable in the \texttt{sys} module. You can
access this list by executing \texttt{import sys}. The length of the list is at least one; when no
script and no arguments are given, \texttt{sys.argv[0]} is an empty string. When the script name is
given as \texttt{'-'} (meaning standard input), \texttt{sys.argv[0]} is set to \texttt{'-'}. When
\texttt{-c} command is used, \texttt{sys.argv[0]} is set to \texttt{'-c'}. When \texttt{-m} module
is used, \texttt{sys.argv[0]} is set to the full name of the located module. Options found after
\texttt{-c} command or \texttt{-m} module are not consumed by the Python interpreter’s option
processing but left in \texttt{sys.argv} for the command or module to handle.

\subsubsection{Interactive Mode}
\begin{itemize}
    \item \emph{primary prompt}: \texttt{>>>}
    \begin{minted}{bash}
    Python 2.7.10 (default, Feb  7 2017, 00:08:15)
    Type "help", "copyright", "credits" or "license" for more information.
    >>>
    \end{minted}
    \item \emph{secondary prompt}: \texttt{...}
    \begin{minted}{bash}
    >>> the_world_is_flat = 1
    >>> if the_world_is_flat:
    ...     print "Be careful not to fall off!"
    ...
    Be careful not to fall off!
    \end{minted}
\end{itemize}

\subsection{The Interpreter and Its Environment}
\subsubsection{Source Code Encoding}
By default, Python source files are treated as encoded in UTF-8.

To declare an encoding other than the default one, a special comment line should be added as the
first line of the file. The syntax is as follows:
\begin{minted}{Python}
    # -*- coding: encoding -*-
\end{minted}
where \emph{encoding} is one of the valid \texttt{codes} supported by Python.










\end{document}
