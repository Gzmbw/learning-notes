%-*- coding: UTF-8 -*-
% Math.tex
%
\documentclass[UTF8]{ctexart}
\usepackage{geometry}
\geometry{a4paper, centering, scale=0.8}
\usepackage{minted}
\usepackage{amsmath}
\usepackage{amssymb} % to use \mathbb

\title{\heiti Chapter 3 Typesetting Mathematical Formulae}
\author{\kaishu Du Ang \\ \texttt{du2ang233@gmail.com} }
\date{\today}

\begin{document}
\maketitle

\tableofcontents

\newpage

本章将简单地介绍如何用 \LaTeX 进行它所擅长的数学排版。如果这一章介绍的数学排版内容无法解决你的问题,不要灰心,使用
\AmS-\LaTeX{} 宏集,一般都能找到答案。

\section{\AmS-\LaTeX{} 宏集}
\AmS-\LaTeX{}宏集是一些用于数学排版的包和类的集合,可以排版出高质量的数学内容。\texttt{amsmath} 宏包是这个宏集的
核心部分,一般用这个就足够了。\AmS-\LaTeX{}是由美国数学学会(American Mathematical Society)提供的对 \LaTeX
原生的数学公式排版的扩展,目前 \TeX{}最近的发行版都默认包含它了。在本章,需要在导言区使用
\mintinline{LaTeX}{\usepackage{amsmath}} 命令引入 \texttt{amsmath} 宏包。

\section{单个方程}
行内公式应该放在一对 \texttt{\$} 之间。

示例代码:
\begin{minted}{LaTeX}
    Add $a$ squared and $b$ squared to get $c$ squared. Or, using
    a more mathematical approach:
    $a^2 + b^2 = c^2$

    \TeX{} is pronounced as $\tau\epsilon\chi$ \\ [5pt]
    100~m$^{3}$ of water \\ [5pt]
    This comes from my $\heartsuit$
\end{minted}

示例输出:
\begin{quote}
    Add $a$ squared and $b$ squared to get $c$ squared. Or, using a more mathematical approach:
    $a^2 + b^2 = c^2$

    \TeX{} is pronounced as $\tau\epsilon\chi$ \\ [5pt]
    100~m$^{3}$ of water \\ [5pt]
    This comes from my $\heartsuit$
\end{quote}

行间公式放在 \mintinline{LaTeX}{\begin{equation}} 和 \mintinline{LaTeX}{\end{equation}} 之间。可以通过
\mintinline{LaTeX}{\labell} 和 \mintinline{LaTeX}{\eqref} 来给公式添加标签和建立引用,用
\mintinline{LaTeX}{\tag} 来给公式指定具体的名字。

示例代码:
\begin{minted}{LaTeX}
    Add $a$ squared and $b$ squared to get $c$ squared. Or, using
    a more mathematical approach:
        \begin{equation}
            a^2 + b^2 = c^2
        \end{equation}
    Einstein says
        \begin{equation}
            E = mc^2 \label{clever}
        \end{equation}
    He didn't say
        \begin{equation}
            1 + 1 = 3 \tag{dumb}
        \end{equation}
    This a reference to \eqref{clever}.
\end{minted}

示例输出:
\begin{quote}
    Add $a$ squared and $b$ squared to get $c$ squared. Or, using
    a more mathematical approach:
        \begin{equation}
            a^2 + b^2 = c^2
        \end{equation}
    Einstein says
        \begin{equation}
            E = mc^2 \label{clever}
        \end{equation}
    He didn't say
        \begin{equation}
            1 + 1 = 3 \tag{dumb}
        \end{equation}
    This a reference to \eqref{clever}.
\end{quote}

如果不想给公式编号,用 \texttt{equation} 的加星版本 \texttt{equation*};或者把公式放在
\mintinline{LaTeX}{\[} 和 \mintinline{LaTeX}{\]} 之间。

示例代码:
\begin{minted}{LaTeX}
    Add $a$ squared and $b$ squared to get $c$ squared. Or, using a more mathematical approach
        \begin{equation*}
            a^2 + b^2 = c^2
        \end{equation*}
    or you can type less for the same effect:
        \[ a^2 + b^2 = c^2 \]
\end{minted}

示例输出:
\begin{quote}
    Add $a$ squared and $b$ squared to get $c$ squared. Or, using a more mathematical approach
        \begin{equation*}
            a^2 + b^2 = c^2
        \end{equation*}
    or you can type less for the same effect:
        \[ a^2 + b^2 = c^2 \]
\end{quote}

虽然 \mintinline{LaTex}{\[} 和 \mintinline{LaTeX}{\]} 很简洁,但是使用时不能像 \texttt{equation} 和
\texttt{equation*} 那样在有编号和无编号之间切换。

注意排版格式中行内公式(text style)和行间公式(display style)的区别。

示例代码:
\begin{minted}{LaTeX}
    This is text style:
    $\lim_{n \to \infty} \sum_{k=1}^n \frac{1}{k^2} = \frac{\pi^2}{6}$.

    And this is display style:
        \begin{equation}
            \lim_{n \to \infty} \sum_{k=1}^n \frac{1}{k^2} = \frac{\pi^2}{6}
        \end{equation}
\end{minted}

示例输出:
\begin{quote}
    This is text style:
    $\lim_{n \to \infty} \sum_{k=1}^n \frac{1}{k^2} = \frac{\pi^2}{6}$.

    And this is display style:
        \begin{equation}
            \lim_{n \to \infty} \sum_{k=1}^n \frac{1}{k^2} = \frac{\pi^2}{6}
        \end{equation}
\end{quote}

在排版行间公式时,可以把一些比较高的公式放在 \mintinline{LaTeX}{\smash} 命令里,让 \LaTeX 忽略这些公式的高度,使
行间距保持不变。

示例代码:
\begin{minted}{LaTeX}
    A $d_{e_{e_{p_{e_r}}}}$ mathematical expression followed \\
    by a $h^{i^{g^{h^{e^r}}}}$ expression. As opposed to a \\
    smashed \smash{$d_{e_{e_{p_{e_r}}}}$} expression followed by a
    \smash{$h^{i^{g^{h^{e^r}}}}$} \\ expression.
\end{minted}

示例输出:
\begin{quote}
    A $d_{e_{e_{p_{e_r}}}}$ mathematical expression followed \\
    by a $h^{i^{g^{h^{e^r}}}}$ expression. As opposed to a \\
    smashed \smash{$d_{e_{e_{p_{e_r}}}}$} expression followed by a
    \smash{$h^{i^{g^{h^{e^r}}}}$} \\ expression.
\end{quote}

\subsection{数学模式(Math Mode)}
当使用 \mintinline{LaTeX}{$} 开启行内公式输入,或是使用 \mintinline{LaTeX}{\[} 命令、\texttt{equation} 环
境时,就进入了数学模式。数学模式相比于文本模式(text mode)有以下特点:
\begin{enumerate}
    \item 大多数的空格、换行都不起作用,数学符号的间隙默认完全由符号的性质(关系符号、运算符等)决定。需要人为引入空
    隙时,使用 \mintinline{LaTeX}{\,}、\mintinline{LaTeX}{\quad} 和 \mintinline{LaTeX}{\qquad} 等命令
    \item 不允许有空行、分段
    \item 所有的字母被当作公式中的变量处理,字母间距与文本模式不一致,也无法生成单词间的空格。如果想在数学公式中输入
    正体的文本,可以使用 \mintinline{LaTeX}{\text{...}} 命令
\end{enumerate}

示例代码:
\begin{minted}{LaTeX}
    $\forall x \in \mathbf{R}: \qquad x^{2} \geq 0$

    $x^{2} \geq 0\qquad \text{for all }x\in\mathbf{R}$
\end{minted}

示例输出:
\begin{quote}
    $\forall x \in \mathbf{R}: \qquad x^{2} \geq 0$

    $x^{2} \geq 0\qquad \text{for all }x\in\mathbf{R}$
\end{quote}

数学家们对应该使用什么符号很挑剔:上面的公式最好使用“blackboard bold”字体,可以使用 \texttt{amssymb} 宏包里的
\mintinline{LaTeX}{\mathbb} 命令来完成。

示例代码:
\begin{minted}{LaTeX}
    $x^{2} \geq 0\qquad \text{for all } x \in \mathbb{R}$
\end{minted}

示例输出:
\begin{quote}
    $x^{2} \geq 0\qquad \text{for all } x \in \mathbb{R}$
\end{quote}




\end{document}
