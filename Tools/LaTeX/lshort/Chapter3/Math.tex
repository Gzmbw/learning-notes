%-*- coding: UTF-8 -*-
% Math.tex
%
\documentclass[UTF8]{ctexart}
\usepackage{geometry}
\geometry{a4paper, centering, scale=0.8}
\usepackage{minted}
\usepackage{amsmath}
\usepackage{amssymb} % to use \mathbb
\usepackage[retainorgcmds]{IEEEtrantools} % to use \IEEEeqnarray
\usepackage{amsthm}

\title{\heiti 第3章 \quad 排版数学公式}
\author{\kaishu Du Ang \\ \texttt{du2ang233@gmail.com} }
\date{\today}

\DeclareMathOperator{\argh}{argh}
\DeclareMathOperator*{\nut}{Nut}

\begin{document}
\maketitle

\tableofcontents

\newpage

本章将简单地介绍如何用 \LaTeX 进行它所擅长的数学排版。如果这一章介绍的数学排版内容无法解决你的问题,不要灰心,使用
\AmS-\LaTeX{} 宏集,一般都能找到答案。

\section{\AmS-\LaTeX{} 宏集}
\AmS-\LaTeX{}宏集是一些用于数学排版的包和类的集合,可以排版出高质量的数学内容。\texttt{amsmath} 宏包是这个宏集的
核心部分,一般用这个就足够了。\AmS-\LaTeX{}是由美国数学学会(American Mathematical Society)提供的对 \LaTeX
原生的数学公式排版的扩展,目前 \TeX{}最近的发行版都默认包含它了。在本章,需要在导言区使用
\mintinline{LaTeX}{\usepackage{amsmath}} 命令引入 \texttt{amsmath} 宏包。

\section{单个方程}
行内公式应该放在一对 \texttt{\$} 之间。

示例代码:
\begin{minted}{LaTeX}
    Add $a$ squared and $b$ squared to get $c$ squared. Or, using
    a more mathematical approach:
    $a^2 + b^2 = c^2$

    \TeX{} is pronounced as $\tau\epsilon\chi$ \\ [5pt]
    100~m$^{3}$ of water \\ [5pt]
    This comes from my $\heartsuit$
\end{minted}

示例输出:
\begin{quote}
    Add $a$ squared and $b$ squared to get $c$ squared. Or, using a more mathematical approach:
    $a^2 + b^2 = c^2$

    \TeX{} is pronounced as $\tau\epsilon\chi$ \\ [5pt]
    100~m$^{3}$ of water \\ [5pt]
    This comes from my $\heartsuit$
\end{quote}

行间公式放在 \mintinline{LaTeX}{\begin{equation}} 和 \mintinline{LaTeX}{\end{equation}} 之间。可以通过
\mintinline{LaTeX}{\labell} 和 \mintinline{LaTeX}{\eqref} 来给公式添加标签和建立引用,用
\mintinline{LaTeX}{\tag} 来给公式指定具体的名字。

示例代码:
\begin{minted}{LaTeX}
    Add $a$ squared and $b$ squared to get $c$ squared. Or, using
    a more mathematical approach:
        \begin{equation}
            a^2 + b^2 = c^2
        \end{equation}
    Einstein says
        \begin{equation}
            E = mc^2 \label{clever}
        \end{equation}
    He didn't say
        \begin{equation}
            1 + 1 = 3 \tag{dumb}
        \end{equation}
    This a reference to \eqref{clever}.
\end{minted}

示例输出:
\begin{quote}
    Add $a$ squared and $b$ squared to get $c$ squared. Or, using
    a more mathematical approach:
        \begin{equation}
            a^2 + b^2 = c^2
        \end{equation}
    Einstein says
        \begin{equation}
            E = mc^2 \label{clever}
        \end{equation}
    He didn't say
        \begin{equation}
            1 + 1 = 3 \tag{dumb}
        \end{equation}
    This a reference to \eqref{clever}.
\end{quote}

如果不想给公式编号,用 \texttt{equation} 的加星版本 \texttt{equation*};或者把公式放在
\mintinline{LaTeX}{\[} 和 \mintinline{LaTeX}{\]} 之间。

示例代码:
\begin{minted}{LaTeX}
    Add $a$ squared and $b$ squared to get $c$ squared. Or, using a more mathematical approach
        \begin{equation*}
            a^2 + b^2 = c^2
        \end{equation*}
    or you can type less for the same effect:
        \[ a^2 + b^2 = c^2 \]
\end{minted}

示例输出:
\begin{quote}
    Add $a$ squared and $b$ squared to get $c$ squared. Or, using a more mathematical approach
        \begin{equation*}
            a^2 + b^2 = c^2
        \end{equation*}
    or you can type less for the same effect:
        \[ a^2 + b^2 = c^2 \]
\end{quote}

虽然 \mintinline{LaTex}{\[} 和 \mintinline{LaTeX}{\]} 很简洁,但是使用时不能像 \texttt{equation} 和
\texttt{equation*} 那样在有编号和无编号之间切换。

注意排版格式中行内公式(text style)和行间公式(display style)的区别。

示例代码:
\begin{minted}{LaTeX}
    This is text style:
    $\lim_{n \to \infty} \sum_{k=1}^n \frac{1}{k^2} = \frac{\pi^2}{6}$.

    And this is display style:
        \begin{equation}
            \lim_{n \to \infty} \sum_{k=1}^n \frac{1}{k^2} = \frac{\pi^2}{6}
        \end{equation}
\end{minted}

示例输出:
\begin{quote}
    This is text style:
    $\lim_{n \to \infty} \sum_{k=1}^n \frac{1}{k^2} = \frac{\pi^2}{6}$.

    And this is display style:
        \begin{equation}
            \lim_{n \to \infty} \sum_{k=1}^n \frac{1}{k^2} = \frac{\pi^2}{6}
        \end{equation}
\end{quote}

在排版行间公式时,可以把一些比较高的公式放在 \mintinline{LaTeX}{\smash} 命令里,让 \LaTeX 忽略这些公式的高度,使
行间距保持不变。

示例代码:
\begin{minted}{LaTeX}
    A $d_{e_{e_{p_{e_r}}}}$ mathematical expression followed \\
    by a $h^{i^{g^{h^{e^r}}}}$ expression. As opposed to a \\
    smashed \smash{$d_{e_{e_{p_{e_r}}}}$} expression followed by a
    \smash{$h^{i^{g^{h^{e^r}}}}$} \\ expression.
\end{minted}

示例输出:
\begin{quote}
    A $d_{e_{e_{p_{e_r}}}}$ mathematical expression followed \\
    by a $h^{i^{g^{h^{e^r}}}}$ expression. As opposed to a \\
    smashed \smash{$d_{e_{e_{p_{e_r}}}}$} expression followed by a
    \smash{$h^{i^{g^{h^{e^r}}}}$} \\ expression.
\end{quote}

\subsection{数学模式(Math Mode)}
当使用 \mintinline{LaTeX}{$} 开启行内公式输入,或是使用 \mintinline{LaTeX}{\[} 命令、\texttt{equation} 环
境时,就进入了数学模式。数学模式相比于文本模式(text mode)有以下特点:
\begin{enumerate}
    \item 大多数的空格、换行都不起作用,数学符号的间隙默认完全由符号的性质(关系符号、运算符等)决定。需要人为引入空
    隙时,使用 \mintinline{LaTeX}{\,}、\mintinline{LaTeX}{\quad} 和 \mintinline{LaTeX}{\qquad} 等命令
    \item 不允许有空行、分段
    \item 所有的字母被当作公式中的变量处理,字母间距与文本模式不一致,也无法生成单词间的空格。如果想在数学公式中输入
    正体的文本,可以使用 \mintinline{LaTeX}{\text{...}} 命令
\end{enumerate}

示例代码:
\begin{minted}{LaTeX}
    $\forall x \in \mathbf{R}: \qquad x^{2} \geq 0$

    $x^{2} \geq 0\qquad \text{for all }x\in\mathbf{R}$
\end{minted}

示例输出:
\begin{quote}
    $\forall x \in \mathbf{R}: \qquad x^{2} \geq 0$

    $x^{2} \geq 0\qquad \text{for all }x\in\mathbf{R}$
\end{quote}

数学家们对应该使用什么符号很挑剔:上面的公式最好使用“blackboard bold”字体,可以使用 \texttt{amssymb} 宏包里的
\mintinline{LaTeX}{\mathbb} 命令来完成。

示例代码:
\begin{minted}{LaTeX}
    $x^{2} \geq 0\qquad \text{for all } x \in \mathbb{R}$
\end{minted}

示例输出:
\begin{quote}
    $x^{2} \geq 0\qquad \text{for all } x \in \mathbb{R}$
\end{quote}

\section{构建数学公式块}
在这一小节中大部分的命令都不需要引入 \texttt{amsmath} 宏包。

\subsection{希腊字母}
小写希腊字母通过 \mintinline{LaTeX}{\alpha}、\mintinline{LaTeX}{\beta}、\mintinline{LaTeX}{\gamma} 等输
入,大写希腊字母通过 \mintinline{LaTeX}{\Gamma}、\mintinline{LaTeX}{\Delta} 等输入。

示例代码:
\begin{minted}{LaTeX}
    $\lambda, \xi, \pi, \theta, \mu, \Phi, \Omega, \Delta$
\end{minted}

示例输出:
\begin{quote}
    $\lambda, \xi, \pi, \theta, \mu, \Phi, \Omega, \Delta$
\end{quote}

\subsection{指数(Exponents),上标(Superscripts),下标(Subscripts)}
指数、上标和下标可以分别通过 \mintinline{LaTeX}{^} 和 \mintinline{LaTeX}{_} 指定。大多数的数学模式命令仅对它之后
的那个字母起作用,所以如果想对多个字母起作用,应该用 \mintinline{LaTeX}|{...}| 括起来。

示例代码:
\begin{minted}{LaTeX}
    $p^3_{ij} \qquad m_\text{Knuth}\qquad \sum_{k=1}^3 k \\[5pt]
    a^x+y \neq a^{x+y}\qquad e^{x^2} \neq {e^x}^2$
\end{minted}

示例输出:
\begin{quote}
    $p^3_{ij} \qquad m_\text{Knuth}\qquad \sum_{k=1}^3 k \\[5pt]
    a^x+y \neq a^{x+y}\qquad e^{x^2} \neq {e^x}^2$
\end{quote}

\subsection{根式}
通过 \mintinline{LaTeX}{\sqrt} 输入平方根(square root);通过 \mintinline{LaTeX}{\sqrt[n]} 输入n次方根。
根号的大小是由 \LaTeX 自动决定的,如果只需要一个符号标记,可以用 \mintinline{LaTeX}{\surd} 命令。

示例代码:
\begin{minted}{LaTeX}
    $\sqrt{x} \Leftrightarrow x^{1/2} \quad \sqrt[3]{2} \quad \sqrt{x^{2} + \sqrt{y}}
    \quad \surd[x^2 + y^2]$
\end{minted}

示例输出:
\begin{quote}
    $\sqrt{x} \Leftrightarrow x^{1/2} \quad \sqrt[3]{2} \quad \sqrt{x^{2} + \sqrt{y}}
    \quad \surd[x^2 + y^2]$
\end{quote}

\subsection{点}
示例代码:
\begin{minted}{LaTeX}
    $\Psi = v_1 \cdot v_2 \cdot \ldots \qquad n! = 1 \cdot 2 \cdots (n-1) \cdot n
    \qquad \vdots \qquad \ddots$
\end{minted}

示例输出:
\begin{quote}
    $\Psi = v_1 \cdot v_2 \cdot \ldots \qquad n! = 1 \cdot 2 \cdots (n-1) \cdot n
    \qquad \vdots \qquad \ddots$
\end{quote}

\subsection{横线}
利用 \mintinline{LaTeX}{\overline} 和 \mintinline{LaTeX}{\underline} 命令来产生上划线和下划线。

示例代码:
\begin{minted}{LaTeX}
    $0.\overline{3} = \underline{\underline{1/3}}$
\end{minted}

示例输出:
\begin{quote}
    $0.\overline{3} = \underline{\underline{1/3}}$
\end{quote}

\subsection{水平花括号}
利用 \mintinline{LaTeX}{\overbrace} 和 \mintinline{LaTeX}{\underbrace} 命令来产生水平的上、下花括号。

示例代码:
\begin{minted}{LaTeX}
    $\underbrace{\overbrace{a+b+c}^6 \cdot \overbrace{d+e+f}^7}_\text{meaning of life} = 42$
\end{minted}

示例输出:
\begin{quote}
    $\underbrace{\overbrace{(a+b+c)}^6 \cdot \overbrace{(d+e+f)}^7}_\text{meaning of life} = 42$
\end{quote}

\subsection{数学重音符号}
示例代码:
\begin{minted}{LaTeX}
    $f(x) = x^2 \qquad f'(x) = 2x \qquad f''(x) = 2\\[5pt]
    \hat{XY} \quad \widehat{XY} \quad \bar{x_0} \quad \bar{x}_0\\[5pt]
    \tilde{a} \quad \widetilde{a}$
\end{minted}

注意:上面示例代码中,\mintinline{LaTeX}{\\} 命令后跟了可选参数 \mintinline{LaTeX}{[5pt]} 来增加额外的行距。

示例输出:
\begin{quote}
    $f(x) = x^2 \qquad f'(x) = 2x \qquad f''(x) = 2\\[5pt]
    \hat{XY} \quad \widehat{XY} \quad \bar{x_0} \quad \bar{x}_0\\[5pt]
    \tilde{a} \quad \widetilde{a}$
\end{quote}

\subsection{向量}
在变量上加箭头来表示向量,可以通过 \mintinline{LaTeX}{\vec} 命令完成。

示例代码:
\begin{minted}{LaTeX}
    $\vec{a} \qquad \vec{AB} \qquad \overrightarrow{AB}$
\end{minted}

示例输出:
\begin{quote}
    $\vec{a} \qquad \vec{AB} \qquad \overrightarrow{AB}$
\end{quote}

\subsection{函数名}
在排版数学公式时,通常函数名不像变量那样用斜体,所以要用命令来进行区别。下面是 \LaTeX 中常用的函数名命令:

\begin{tabular}{lllllll}
    \mintinline{LaTeX}{\arccos} & \mintinline{LaTeX}{\cos} & \mintinline{LaTeX}{\csc}
        & \mintinline{LaTeX}{\exp} & \mintinline{LaTeX}{\ker} & \mintinline{LaTeX}{\limsup} \\
    \mintinline{LaTeX}{\arcsin} & \mintinline{LaTeX}{\cosh} & \mintinline{LaTeX}{\deg}
        & \mintinline{LaTeX}{\gcd} & \mintinline{LaTeX}{\lg} & \mintinline{LaTeX}{\ln} \\
    \mintinline{LaTeX}{\arctan} & \mintinline{LaTeX}{\cot} & \mintinline{LaTeX}{\det}
        & \mintinline{LaTeX}{\hom} & \mintinline{LaTeX}{\lim} & \mintinline{LaTeX}{\log} \\
    \mintinline{LaTeX}{\arg} & \mintinline{LaTeX}{\coth} & \mintinline{LaTeX}{\dim}
        & \mintinline{LaTeX}{\inf} & \mintinline{LaTeX}{\liminf} & \mintinline{LaTeX}{\max} \\
    \mintinline{LaTeX}{\sinh} & \mintinline{LaTeX}{\sup} & \mintinline{LaTeX}{\tan}
        & \mintinline{LaTeX}{\tanh} & \mintinline{LaTeX}{\min} & \mintinline{LaTeX}{\Pr} \\
    \mintinline{LaTeX}{\sec} & \mintinline{LaTeX}{\sin} & & & &
\end{tabular}

对于上面没有列出的一些函数,可以在导言区用 \mintinline{LaTeX}{\DeclareMathOperator} 或其加星版本的命令来定义。
加星和不加星版本主要是在显示为行内公式还是行间公式上有区别。
\footnote{MathJax: \textbackslash DeclareMathOperator and \textbackslash operatorname
vs starred versions: http://mathb.in/13571}

示例代码:
\begin{minted}{LaTeX}
    \begin{equation*}
        \lim_{x \rightarrow 0} \frac{\sin x}{x} = 1
    \end{equation*}

    % \DeclareMathOperator{\argh}{argh}
    % \DeclareMathOperator*{\nut}{Nut}
    \begin{equation*}
        3\argh = 2\nut_{x=1}
    \end{equation*}
\end{minted}

示例输出:
\begin{quote}
    \begin{equation*}
        \lim_{x \rightarrow 0} \frac{\sin x}{x} = 1
    \end{equation*}

    % \DeclareMathOperator{\argh}{argh}
    % \DeclareMathOperator*{\nut}{Nut}
    \begin{equation*}
        3\argh = 2\nut_{x=1}
    \end{equation*}
\end{quote}

\subsection{取模函数}
取模函数有两种命令:\mintinline{LaTeX}{\bmod} 和 \mintinline{LaTeX}{\pmod}。

示例代码:
\begin{minted}{LaTeX}
    $a\bmod b \\
     x\equiv a \pmod{b}$
\end{minted}

示例输出:
\begin{quote}
    $a\bmod b \\
     x\equiv a \pmod{b}$
\end{quote}

\subsection{分式和偏导}
可以通过 \mintinline{LaTeX}{\frac{...}{...}} 命令来排版分式。在行内公式中,分式会收缩以和所在行保持一致。可以通
过 \mintinline{LaTeX}{\tfrac} 命令在行间公式中将分式强制显示为行内模式,同样地,也可以通过
\mintinline{LaTeX}{\dfrac} 命令在行内公式中将分式强制显示为行间模式。一般,在行间公式中更经常使用斜杠的形式来表示
比较短的分式,如1/2。

示例代码:
\begin{minted}{LaTeX}
    In display style:
    \begin{equation*}
        3/8 \qquad \frac{3}{8} \qquad \tfrac{3}{8}
    \end{equation*}

    In text style:
    $1\frac{1}{2}$~hours \qquad $1\dfrac{1}{2}$~hours
\end{minted}

示例输出:
\begin{quote}
    In display style:
    \begin{equation*}
        3/8 \qquad \frac{3}{8} \qquad \tfrac{3}{8}
    \end{equation*}

    In text style:
    $1\frac{1}{2}$~hours \qquad $1\dfrac{1}{2}$~hours
\end{quote}

使用 \mintinline{LaTeX}{\partial} 命令来输入偏导符号。

示例代码:
\begin{minted}{LaTeX}
    \begin{equation*}
        \sqrt{\frac{x^2}{k+1}} \qquad x^\frac{2}{k+1} \qquad
        \frac{\partial^2f}{\partial x^2}
    \end{equation*}
\end{minted}

示例输出:
\begin{quote}
    \begin{equation*}
        \sqrt{\frac{x^2}{k+1}} \qquad x^\frac{2}{k+1} \qquad
        \frac{\partial^2f}{\partial x^2}
    \end{equation*}
\end{quote}

\subsection{二项式系数和符号堆叠}
可以通过 \texttt{amsmath} 宏包里的 \mintinline{LaTeX}{\binom} 命令来排版二项式系数
(binomial coefficients)。

示例代码:
\begin{minted}{LaTeX}
    Pascal's rule is
    \begin{equation*}
        \binom{n}{k} = \binom{n-1}{k} + \binom{n-1}{k-1}
    \end{equation*}
\end{minted}

示例输出:
\begin{quote}
    Pascal's rule is
    \begin{equation*}
        \binom{n}{k} = \binom{n-1}{k} + \binom{n-1}{k-1}
    \end{equation*}
\end{quote}

对于一些二元关系,有时候可能需要对符号进行堆叠,这时可以使用 \mintinline{LaTeX}{\stackrel{#1}{#2}} 命令。其中
\mintinline{LaTeX}{#1} 会变成上标大小,\mintinline{LaTeX}{#2} 会保持原来的大小和位置。

示例代码:
\begin{minted}{LaTeX}
    \begin{equation*}
        f_n(x) \stackrel{*}{\approx} 1
    \end{equation*}
\end{minted}

示例输出:
\begin{quote}
    \begin{equation*}
        f_n(x) \stackrel{*}{\approx} 1
    \end{equation*}
\end{quote}

\subsection{积分运算符、求和运算符、乘积运算符}
分别用 \mintinline{LaTeX}{\int}、\mintinline{LaTeX}{\sum}、\mintinline{LaTeX}{\prod} 命令来输入积分运算
符、求和运算符和乘积运算符。

示例代码:
\begin{minted}{LaTeX}
    \begin{equation*}
        \sum_{i=1}^n \qquad \int_0^{\frac{\pi}{2}} \qquad \prod_\epsilon
    \end{equation*}
\end{minted}

示例输出:
\begin{quote}
    \begin{equation*}
        \sum_{i=1}^n \qquad \int_0^{\frac{\pi}{2}} \qquad \prod_\epsilon
    \end{equation*}
\end{quote}

在更复杂的表达式中,可以通过 \texttt{amsmath} 宏包中的 \mintinline{LaTeX}{\substack} 命令来更好地控制索引的位
置。

示例代码:
\begin{minted}{LaTeX}
    \begin{equation*}
        \sum^n_{\substack{0<i<n \\ j\subseteq i}} P(i,j) = Q(i,j)
    \end{equation*}
\end{minted}

示例输出:
\begin{quote}
    \begin{equation*}
        \sum^n_{\substack{0<i<n \\ j\subseteq i}} P(i,j) = Q(i,j)
    \end{equation*}
\end{quote}

\LaTeX 可以输入各种括号(bracketing)和分隔符(delimiters)。圆括号和方括号可以通过各自的键位输入,花括号需要使用
\mintinline{LaTeX}|\{| 命令,其他的分隔符都要用特殊的命令(如 \mintinline{LaTeX}{\updownarrow})。

示例代码:
\begin{minted}{LaTeX}
    \begin{equation*}
        {a,b,c} \neq \{a,b,c\}
    \end{equation*}
\end{minted}

示例输出:
\begin{quote}
    \begin{equation*}
        {a,b,c} \neq \{a,b,c\}
    \end{equation*}
\end{quote}

把 \mintinline{LaTeX}|\left| 命令放在左分隔符前面,把 \mintinline{LaTeX}|\right| 放在右分隔符前面,这样
\LaTeX 就会自动地调整分隔符的大小。如果不想要右半个分隔符,使用不可见的 \mintinline{LaTeX}|\right.| 命令。

示例代码:
\begin{minted}{LaTeX}
    \begin{equation*}
        1 + \left(\frac{1}{1-x^{2}} \right)^3 \qquad
        \left. \ddagger \frac{~}{~}\right)
    \end{equation*}
\end{minted}

示例输出:
\begin{quote}
    \begin{equation*}
        1 + \left(\frac{1}{1-x^{2}} \right)^3 \qquad
        \left. \ddagger \frac{~}{~}\right)
    \end{equation*}
\end{quote}

在有些情况下,需要通过 \mintinline{LaTeX}|\big|、\mintinline{LaTeX}|\Big|、\mintinline{LaTeX}|\bigg|、
\mintinline{LaTeX}|\Bigg| 来手动指定分隔符的大小。

示例代码:
\begin{minted}{LaTeX}
    $\Big((x+1)(x-1)\Big)^{2}$ \\
    $\big( \Big( \bigg( \Bigg( \quad
    \big\} \Big\} \bigg\} \Bigg\} \quad
    \big\| \Big\| \bigg\| \Bigg\| \quad
    \big\Downarrow \Big\Downarrow
    \bigg\Downarrow \Bigg\Downarrow$
\end{minted}

示例输出:
\begin{quote}
    $\Big((x+1)(x-1)\Big)^{2}$ \\
    $\big( \Big( \bigg( \Bigg( \quad
    \big\} \Big\} \bigg\} \Bigg\} \quad
    \big\| \Big\| \bigg\| \Bigg\| \quad
    \big\Downarrow \Big\Downarrow
    \bigg\Downarrow \Bigg\Downarrow$
\end{quote}

\section{长公式折行}
有时候一个等式或方程太长了,有必要使其折行。但是折行会降低等式的可读性,为了提高可读性,在折行时可遵循以下规则:

\begin{enumerate}
    \item 应该在等号或者某个操作符前折行
    \item 优先在等号前而不是操作符前折行
    \item 优先在加号、减号而不是乘号前折行
    \item 避免其他的折行方式
\end{enumerate}

实现这种折行的最简单方式是使用 \texttt{amsmath} 宏包中的 \texttt{multline} 环境。

示例代码:
\begin{minted}{LaTeX}
    \begin{multline}
        a + b + c + d + e + f + g + h + i + j + k + l + m + n + o + p + q
        \\
        = r + s + t + u + v + w + x + y + z
    \end{multline}
\end{minted}

示例输出:
\begin{quote}
    \begin{multline}
        a + b + c + d + e + f + g + h + i + j + k + l + m + n + o + p + q
        \\
        = r + s + t + u + v + w + x + y + z
    \end{multline}
\end{quote}

与之相比,\texttt{equation} 环境可能会引入任意的一个或多个折行。而使用 \texttt{multline} 环境时,仅在需要折行
的地方使用 \mintinline{LaTeX}{\\} 命令进行折行。和 \texttt{equation*} 类似,\texttt{multline*} 不对公式进
行编号。

示例代码:
\begin{minted}{LaTeX}
    \begin{equation}
        a = b + c + d + e + f + g + h + i + j + k + l + m + n + o + p + q
        + r + s + t + u + v + w + x + y + z + aa + bb + cc + dd + ee + ff
        \label{eq:equation_too_long}
    \end{equation}

    \begin{multline}
        a = b + c + d + e + f + g + h + i + j + k + l + m + n + o + p + q \\
        + r + s + t + u + v + w + x + y + z + aa + bb + cc + dd + ee + ff
        \label{eq:equation_too_long_multl}
    \end{multline}
\end{minted}

示例输出:
\begin{quote}
    \begin{equation}
        a = b + c + d + e + f + g + h + i + j + k + l + m + n + o + p + q
        + r + s + t + u + v + w + x + y + z + aa + bb + cc + dd + ee + ff
        \label{eq:equation_too_long}
    \end{equation}

    \begin{multline}
        a = b + c + d + e + f + g + h + i + j + k + l + m + n + o + p + q \\
        + r + s + t + u + v + w + x + y + z + aa + bb + cc + dd + ee + ff
        \label{eq:equation_too_long_multl}
    \end{multline}
\end{quote}

比较公式~\ref{eq:equation_too_long} 和公式~\ref{eq:equation_too_long_multl},由于等号右边很长,
公式~\ref{eq:equation_too_long_multl} 的输出更好一些,但是违反了等号优先于加号、减号的规则。更好的解决方案是
使用 \texttt{IEEEeqnarray} 环境,将在第~\ref{sec:mult_eq} 节讨论。

\section{多行公式}
\label{sec:mult_eq}
在遇到一些公式中有连等的情况时,我们希望在竖直方向上等号能够对齐。为了应对这个问题,先看一些常用的方法以及它们的不足之
处。

\subsection{传统命令存在的问题}
可以使用 \texttt{align} 环境来实现等号的对齐。

示例代码:
\begin{minted}{LaTeX}
    \begin{align}
        a & = b + c \\
        & = d + e
    \end{align}
\end{minted}

示例输出:
\begin{quote}
    \begin{align}
        a & = b + c \\
        & = d + e
    \end{align}
\end{quote}

但是在某一行公式过长时,使用 \texttt{align} 环境的输出会出现问题。如下面的示例,\texttt{+ v} 应该和 \texttt{d}
对齐,而不是和上面的等号对齐。可以通过在它前面加一些空格(\mintinline{LaTeX}|\hspace{...}|)来实现我们想要的效
果,但是这不够准确。

示例代码:
\begin{minted}{LaTeX}
    \begin{align}
        a & = b + c \\
        & = d + e + f + g + h + i + j + k + l + m + n + o + p + q + r + s + t + u \nonumber \\
        & + v + w + x + y + z + aa + bb + cc \\
        & = dd + ee + ff + gg + hh + ii
    \end{align}
\end{minted}

示例输出:
\begin{quote}
    \begin{align}
        a & = b + c \\
        & = d + e + f + g + h + i + j + k + l + m + n + o + p + q + r + s + t + u \nonumber \\
        & + v + w + x + y + z + aa + bb + cc \\
        & = dd + ee + ff + gg + hh + ii
    \end{align}
\end{quote}

\texttt{eqnarray} 环境提供了一种更好的解决方案。

示例代码:
\begin{minted}{LaTeX}
    \begin{eqnarray}
        a & = & b + c \\
        & = & d + e + f + g + h + i + j + k + l + m + n + o + p + q + r + s + t + u \nonumber \\
        && +\: v + w + x + y + z + aa + bb + cc \\
        & = & dd + ee + ff + gg + hh + ii
    \end{eqnarray}
\end{minted}

示例输出:
\begin{quote}
    \begin{eqnarray}
        a & = & b + c \\
        & = & d + e + f + g + h + i + j + k + l + m + n + o + p + q + r + s + t + u \nonumber \\
        && +\: v + w + x + y + z + aa + bb + cc \\
        & = & dd + ee + ff + gg + hh + ii
    \end{eqnarray}
\end{quote}

但使用 \texttt{eqnarray} 环境仍不是最优的解决方案。等号两边的空太大了,而且和 \texttt{multline} 环境、
\texttt{equation} 环境中的大小不一样。而且有些时候即使左边有足够的空间,右边还是会和公式编号重叠。

示例代码:
\begin{minted}{LaTeX}
    \begin{eqnarray}
        a & = & a = a
    \end{eqnarray}

    \begin{eqnarray}
        a & = & b + c
        \\
        & = & d + e + f + g + h^2 + i^2 + j + k + l + m + n + o + p + q + r + s + t + u + v
         + w + x + y + z
        \label{eq:faultyeqnarray}
    \end{eqnarray}
\end{minted}

示例输出:
\begin{quote}
    \begin{eqnarray}
        a & = & a = a
    \end{eqnarray}

    \begin{eqnarray}
        a & = & b + c
        \\
        & = & d + e + f + g + h^2 + i^2 + j + k + l + m + n + o + p + q + r + s + t + u + v
         + w + x + y + z
        \label{eq:faultyeqnarray}
    \end{eqnarray}
\end{quote}

\texttt{eqnarray} 环境提供了 \mintinline{LaTeX}{\lefteqn} 命令,可以在公式的等号左边部分过长时使用。但是这样并
不是最优的解决方案,而且等号右边部分过短时公式可能不会居中。

示例代码:
\begin{minted}{LaTeX}
    \begin{eqnarray}
        \lefteqn{a + b + c + d + e + f + g + h} \nonumber \\
        & = & i + j + k + l + m \\
        & = & n + o + p + q + r + s
    \end{eqnarray}

    \begin{eqnarray}
        \lefteqn{a + b + c + d + e + f + g + h} \nonumber \\
        & = & i + j
    \end{eqnarray}
\end{minted}

示例输出:
\begin{quote}
    \begin{eqnarray}
        \lefteqn{a + b + c + d + e + f + g + h} \nonumber \\
        & = & i + j + k + l + m \\
        & = & n + o + p + q + r + s
    \end{eqnarray}

    \begin{eqnarray}
        \lefteqn{a + b + c + d + e + f + g + h} \nonumber \\
        & = & i + j
    \end{eqnarray}
\end{quote}

\subsection{\texttt{IEEEeqnarray} 环境}
为了使用 \texttt{IEEEeqnarray} 环境,需要在导言区使用一下命令引入 \texttt{IEEEtrantools} 宏包:\\
\mintinline{LaTeX}|\usepackage[retainorgcmds]{IEEEtrantools}|。

\texttt{IEEEeqnarray} 的一个优点是可以指定公式排列的列数。通常情况下,指定参数为 \texttt{\{rCl\}},表示总共
有三列,第一列右对齐,中间列居中(和小写字母c相比,大写字母C还会使其左右两边留有空隙),第三列左对齐。

示例代码:
\begin{minted}{LaTeX}
    \begin{IEEEeqnarray}{rCl}
        a & = & b + c \\
        & = & d + e + f + g + h + i + j + k \nonumber \\
        && \negmedspace {} + l + m + n + o \\
        & = & p + q + r + s
    \end{IEEEeqnarray}
\end{minted}

示例输出:
\begin{quote}
    \begin{IEEEeqnarray}{rCl}
        a & = & b + c \\
        & = & d + e + f + g + h + i + j + k \nonumber \\
        && \negmedspace {} + l + m + n + o \\
        & = & p + q + r + s
    \end{IEEEeqnarray}
\end{quote}

可以指定任意多列:\texttt{\{c\}} 表示只有居中的一列;\texttt{\{rCll\}} 会增加左对齐的第四列,一般用于添加注
释。

\subsection{\texttt{IEEEeqnarray} 环境的一般用法}
如果某行公式会和公式编号重叠,可以在对应公式行后使用 \mintinline{LaTeX}{\IEEEeqnarraynumspace} 命令来解决。

示例代码:
\begin{minted}{LaTeX}
    \begin{IEEEeqnarray}{rCl}
        a & = & b + c \\
        & = & d + e + f + g + h^2 + i^2 + j + k + l + m + n + o + p + q + r + s + t + u + v
         + w + x + y + z
    \end{IEEEeqnarray}

    \begin{IEEEeqnarray}{rCl}
        a & = & b + c \\
        & = & d + e + f + g + h^2 + i^2 + j + k + l + m + n + o + p + q + r + s + t + u + v
         + w + x + y + z \IEEEeqnarraynumspace
    \end{IEEEeqnarray}
\end{minted}

示例输出:
\begin{quote}
    \begin{IEEEeqnarray}{rCl}
        a & = & b + c \\
        & = & d + e + f + g + h^2 + i^2 + j + k + l + m + n + o + p + q + r + s + t + u + v
         + w + x + y + z
    \end{IEEEeqnarray}

    \begin{IEEEeqnarray}{rCl}
        a & = & b + c \\
        & = & d + e + f + g + h^2 + i^2 + j + k + l + m + n + o + p + q + r + s + t + u + v
         + w + x + y + z \IEEEeqnarraynumspace
    \end{IEEEeqnarray}
\end{quote}

如果等号左边部分过长,\texttt{IEEEeqnarray} 提供了 \mintinline{LaTeX}{\IEEEeqnarraymulticol} 命令来取
代有缺陷的 \mintinline{LaTeX}{\lefteqn} 命令。

示例代码:
\begin{minted}{LaTeX}
    \begin{IEEEeqnarray}{rCl}
        \IEEEeqnarraymulticol{3}{l}{
            a + b + c + d + e + f + g + h
        } \nonumber \\ \quad
        & = & i + j \\
        & = & k + l + m
    \end{IEEEeqnarray}

    \begin{IEEEeqnarray}{rCl}
        \IEEEeqnarraymulticol{3}{l}{
            a + b + c + d + e + f + g + h
        } \nonumber \\ \qquad \qquad
        & = & i + j \\
        & = & k + l + m
    \end{IEEEeqnarray}
\end{minted}

示例输出:
\begin{quote}
    \begin{IEEEeqnarray}{rCl}
        \IEEEeqnarraymulticol{3}{l}{
            a + b + c + d + e + f + g + h
        } \nonumber \\ \quad
        & = & i + j \\
        & = & k + l + m
    \end{IEEEeqnarray}

    \begin{IEEEeqnarray}{rCl}
        \IEEEeqnarraymulticol{3}{l}{
            a + b + c + d + e + f + g + h
        } \nonumber \\ \qquad \qquad
        & = & i + j \\
        & = & k + l + m
    \end{IEEEeqnarray}
\end{quote}

\mintinline{LaTeX}{\IEEEeqnarraymulticol} 的用法和 \texttt{tabular} 环境中
\mintinline{LaTeX}{\multicolumns} 命令的用法是相同的。第一个参数 \texttt{\{3\}} 指定三列合并为一列,
\texttt{\{l\}} 参数指定其左对齐。可以通过 \mintinline{LaTeX}{\quad} 等命令可以方便地控制等号的缩进。

如果公式折成了两行或多行,\LaTeX 会把首个 $+$ 或者 $-$ 作为一个标识符,而不是运算符。因此需要在运算符和操作数之
间插入额外的空格。

示例代码:
\begin{minted}{LaTeX}
    \begin{IEEEeqnarray}{rCl}
        a & = & b + c \\
        & = & d + e + f + g + h + i + j + k \nonumber \\
        && + l + m + n + o \\
        & = & p + q + r + s
    \end{IEEEeqnarray}

    \begin{IEEEeqnarray}{rCl}
        a & = & b + c \\
        & = & d + e + f + g + h + i + j + k \nonumber \\
        && \negmedspace {} + l + m + n + o \\ % negative medium space
        & = & p + q + r + s
    \end{IEEEeqnarray}
\end{minted}

示例输出:
\begin{quote}
    \begin{IEEEeqnarray}{rCl}
        a & = & b + c \\
        & = & d + e + f + g + h + i + j + k \nonumber \\
        && + l + m + n + o \\
        & = & p + q + r + s
    \end{IEEEeqnarray}

    \begin{IEEEeqnarray}{rCl}
        a & = & b + c \\
        & = & d + e + f + g + h + i + j + k \nonumber \\
        && \negmedspace {} + l + m + n + o \\
        & = & p + q + r + s
    \end{IEEEeqnarray}
\end{quote}

\texttt{IEEEeqnarray} 环境也有加星版本,会省略所有的公式编号,但可以使用
\mintinline{LaTeX}{\IEEEyesnumber} 和 \mintinline{LaTeX}{\IEEEyessubnumber}
命令来显示某行公式的编号或子编号。

示例代码:
\begin{minted}{LaTeX}
    \begin{IEEEeqnarray*}{rCl}
        a & = & b + c \\
        & = & d + e \IEEEyesnumber \\
        & = & f + g
    \end{IEEEeqnarray*}

    \begin{IEEEeqnarray}{rCl}
        a & = & b + c \IEEEyessubnumber \\
        & = & d + e \nonumber \\
        & = & f + g \IEEEyessubnumber
    \end{IEEEeqnarray}
\end{minted}

示例输出:
\begin{quote}
    \begin{IEEEeqnarray*}{rCl}
        a & = & b + c \\
        & = & d + e \IEEEyesnumber \\
        & = & f + g
    \end{IEEEeqnarray*}

    \begin{IEEEeqnarray}{rCl}
        a & = & b + c \IEEEyessubnumber \\
        & = & d + e \nonumber \\
        & = & f + g \IEEEyessubnumber
    \end{IEEEeqnarray}
\end{quote}

\section{数组和矩阵}
可以通过 \texttt{array} 环境来排版数组,它的工作方式和 \texttt{tabular} 环境类似。

示例代码:
\begin{minted}{LaTeX}
    \begin{equation*}
        \mathbf{X} = \left(
            \begin{array}{ccc}
                x_1 & x_2 & \ldots \\
                x_3 & x_4 & \ldots \\
                \vdots & \vdots & \ddots
            \end{array} \right)
    \end{equation*}
\end{minted}

示例输出:
\begin{quote}
    \begin{equation*}
        \mathbf{X} = \left(
            \begin{array}{ccc}
                x_1 & x_2 & \ldots \\
                x_3 & x_4 & \ldots \\
                \vdots & \vdots & \ddots
            \end{array} \right)
    \end{equation*}
\end{quote}

通过将 \texttt{.} 作为一个不可见的右分隔符,\texttt{array} 环境也可以用于排版分段函数(piecewise functions)。
\texttt{amsmath} 宏包中的 \texttt{cases} 环境也可以实现类似的效果,且语法更为简洁。

示例代码:
\begin{minted}{LaTeX}
    \begin{equation*}
        |x| = \left\{
            \begin{array}{rl}
                -x & \text{if } x < 0, \\
                0 & \text{if } x = 0, \\
                x & \text{if } x > 0.
            \end{array} \right.
    \end{equation*}

    \begin{equation*}
        |x| =
        \begin{cases}
            -x & \text{if } x < 0, \\
            0 & \text{if } x = 0, \\
            x & \text{if } x > 0.
        \end{cases}
    \end{equation*}
\end{minted}

示例输出:
\begin{quote}
    \begin{equation*}
        |x| = \left\{
            \begin{array}{rl}
                -x & \text{if } x < 0, \\
                0 & \text{if } x = 0, \\
                x & \text{if } x > 0.
            \end{array} \right.
    \end{equation*}

    \begin{equation*}
        |x| =
        \begin{cases}
            -x & \text{if } x < 0, \\
            0 & \text{if } x = 0, \\
            x & \text{if } x > 0.
        \end{cases}
    \end{equation*}
\end{quote}

可以通过 \texttt{array} 环境来排版矩阵,但是 \texttt{amsmath} 提供了一系列的 \texttt{matrix} 环境,可以更好
地排版矩阵。共有六种不同的分隔符:\texttt{matrix}(无)、\texttt{pmatrix}(()、\texttt{bmatrix}([)、
\texttt{Bmatrix}(\{)、\texttt{vmatrix}(|) 和 \texttt{Vmatrix}(||)。和 \texttt{array} 环境不同,
\texttt{matrix} 环境不需要指定列数,最大列数为10。

\section{数学模式中的空格}
如果 \LaTeX 数学公式中的空格不能满足要求,可以通过插入一些特殊的命令进行调整。\mintinline{LaTeX}{\,} 命令对应
$\frac{3}{18}$ 个quad;\mintinline{LaTeX}{\:} 命令对应 $\frac{4}{18}$ 个 quad;
\mintinline{LaTeX}{\;} 命令对应 $\frac{5}{18}$ 个quad;转移空格字符(escaped space character)
\verb*|\ | 会产生一个和单词间距大小相当的空格;\mintinline{LaTeX}{\quad} 会产生一个当前字体下‘M’字母宽度的空格,
\mintinline{LaTeX}{\qquad} 相当于两个 \mintinline{LaTeX}{\quad};\mintinline{LaTeX}{\!}
命令会产生一个$-\frac{3}{18}$ 个quad。

示例代码:
\begin{minted}{LaTeX}
    \begin{equation*}
        \int_1^2 \ln x \mathrm{d}x
        \qquad
        \int_1^2 \ln x \, \mathrm{d}x
    \end{equation*}
\end{minted}

示例输出:
\begin{quote}
    \begin{equation*}
        \int_1^2 \ln x \mathrm{d}x
        \qquad
        \int_1^2 \ln x \, \mathrm{d}x
    \end{equation*}
\end{quote}

注意微分运算中的字母‘d’一般用罗马字体。下面的例子会定义一个新命令 \mintinline{LaTeX}{\ud}(upright d),可以实现
相同的效果。

示例代码:
\begin{minted}{LaTeX}
    \newcommand{\ud}{\, \mathrm{d}}

    \begin{IEEEeqnarray*}{c}
        \int\int f(x)g(y) \ud x \ud y \\
        \int\!\!\!\int f(x)g(y) \ud x \ud y \\
        \iint f(x)g(y) \ud x \ud y
    \end{IEEEeqnarray*}
\end{minted}

示例输出:
\begin{quote}
    \newcommand{\ud}{\, \mathrm{d}}

    \begin{IEEEeqnarray*}{c}
        \int\int f(x)g(y) \ud x \ud y \\
        \int\!\!\!\int f(x)g(y) \ud x \ud y \\
        \iint f(x)g(y) \ud x \ud y
    \end{IEEEeqnarray*}
\end{quote}

\subsection{幽灵(Phantoms)}
\mintinline{LaTeX}{\phantom} 命令可以为其中的内容保留位置,但这些内容又不会在最终的输出中出现。下面的例子可以帮助
我们很好地理解它的用途。

示例代码:
\begin{minted}{LaTeX}
    \begin{equation*}
        {}^{14}_{6}\text{C}
        \qquad \text{versus} \qquad
        {}^{14}_{\phantom{1}6}\text{C}
    \end{equation*}
\end{minted}

示例输出:
\begin{quote}
    \begin{equation*}
        {}^{14}_{6}\text{C}
        \qquad \text{versus} \qquad
        {}^{14}_{\phantom{1}6}\text{C}
    \end{equation*}
\end{quote}

\texttt{mhchem} 宏包可以更方便地输入同位素和化学方程式。

\section{折腾数学字体}
示例代码:
\begin{minted}{LaTeX}
    $\Re \qquad \mathcal{R} \qquad \mathfrak{R} \qquad \mathbb{R} \qquad $
\end{minted}

示例输出:
\begin{quote}
    $\Re \qquad \mathcal{R} \qquad \mathfrak{R} \qquad \mathbb{R} \qquad $
\end{quote}

上面的后两种字体命令需要用到 \texttt{amssymb} 或 \texttt{amsfonts} 宏包。

有时候需要给 \LaTeX 指定合适的字体大小。在数学模式中,可以通过以下的命令进行设定:\\
\mintinline{LaTeX}{\displaystyle}($\displaystyle 123$),
\mintinline{LaTeX}{\textstyle}($\textstyle 123$),
\mintinline{LaTeX}{\scriptstyle}($\scriptstyle 123$),
\mintinline{LaTeX}{\scriptscriptsytle}($\scriptscriptstyle 123$)。

如果 $\sum$ 符号放在分式中,它会默认排版成文本模式,所以可以根据需要用 \mintinline{LaTeX}{\displaystyle}
进行指定。

示例代码:
\begin{minted}{LaTeX}
    \begin{equation*}
        P = \frac{\displaystyle{\sum_{i=1}^n (x_i-x)(y_i-y)}}
            {\displaystyle{\left[\sum_{i=1}^n(x_i-x)^2 \sum_{i=1}^n(y_i-y)^2 \right]^{1/2}}}
    \end{equation*}
\end{minted}

示例输出:
\begin{quote}
    \begin{equation*}
        P = \frac{\displaystyle{\sum_{i=1}^n (x_i-x)(y_i-y)}}
            {\displaystyle{\left[\sum_{i=1}^n(x_i-x)^2 \sum_{i=1}^n(y_i-y)^2 \right]^{1/2}}}
    \end{equation*}
\end{quote}

\subsection{粗体符号}
在 \LaTeX 中打印粗体符号有一些困难。\mintinline{LaTeX}{\mathbf} 命令可以打印粗体的的英文字母,但是打印出的是罗马
字体的,而数学公式中的字母大都是斜体的,而且该命令对希腊字母无效。\mintinline{LaTeX}{\boldmath} 命令可以打印粗体
的英文字母和希腊字母,但是它不能用在数学公式中,只能包裹在数学公式外面。

\texttt{amsbsy} 宏包(包含于 \texttt{amsmath})和 \texttt{bm} 宏包(包含于 \texttt{tools})中都有一个
\mintinline{LaTeX}{boldsymbol} 命令,可以比较方便地打印粗体符号。

示例代码:
\begin{minted}{LaTeX}
    $\mu, M \qquad \mathbf{\mu}, \mathbf{M}$
    \qquad \boldmath{$\mu, M$}
    $\qquad \boldsymbol{\mu}, \boldsymbol{M}$
\end{minted}

\begin{quote}
    $\mu, M \qquad \mathbf{\mu}, \mathbf{M}$
    \qquad \boldmath{$\mu, M$}
    $\qquad \boldsymbol{\mu}, \boldsymbol{M}$
\end{quote}

\section{定理(theorems),引理(Lemmas)……}
在撰写数学文档的时候,可能需要排版引理、定义、公理等。可以通过下面的命令来完成:
\begin{minted}{LaTeX}
    \newtheorem{name}[counter]{text}[section]
\end{minted}

其中,\emph{name} 参数是我们定义定理的名称,作为一个环境来使用;\emph{text} 参数是定理真正的名字,会显示在
最终的文档中。方括号中的参数是可选的,定理的序号由两个可选参数之一决定,它们不能同时使用。\emph{section} 为章节
名称,使定理序号成为章节的下一级序号。\emph{counter} 可以为用 \mintinline{LaTeX}{\newcounter} 自定义的计数器
名称,定理序号由这个计数器管理;也可以为前面定义的“theorem”的名称。如果两个参数都不用的话,则使用一个默认的计数器。

例如,我们用 \mintinline{LaTeX}{\newtheorem{mythm}{My Theorem}[section]} 命令定义了一个
\texttt{mythem} 环境,然后就可以用它来排版定理。定理带有一个可选参数,可以用于注明定理的名称。

示例代码:
\begin{minted}{LaTeX}
    \newtheorem{mythm}{My Theorem}[section]

    \begin{mythm}
        \label{thm:light}
        The light speed in vaccum is $299,792,458\,\mathrm{m/s}$.
    \end{mythm}

    \begin{mythm}[Energy]
        The relationship of energy, momentum and mass is
            \[E^2 = m_0^2 c^4 + p^2 c^2\]
        where $c$ is the light speed described in theorem \ref{thm:light}.
    \end{mythm}
\end{minted}

示例输出:
\begin{quote}
    \newtheorem{mythm}{My Theorem}[section] % custom theorem

    \begin{mythm}
        \label{thm:light}
        The light speed in vaccum is $299,792,458\,\mathrm{m/s}$.
    \end{mythm}

    \begin{mythm}[Energy]
        The relationship of energy, momentum and mass is
            \[E^2 = m_0^2 c^4 + p^2 c^2\]
        where $c$ is the light speed described in theorem \ref{thm:light}.
    \end{mythm}
\end{quote}

\texttt{amsthm} 宏包提供了 \mintinline{LaTeX}|\theoremstyle{style}| 命令,可以定义不同的 theorem 类型,包括
\texttt{definition}、\texttt{plain}、\texttt{remark} 三种。

示例代码:
\begin{minted}{LaTeX}
    \theoremstyle{definition}   \newtheorem{law}{Law}
    \theoremstyle{plain}        \newtheorem{jury}[law]{Jury}
    \theoremstyle{remark}       \newtheorem*{marg}{Margaret}

    \begin{law}
        \label{law:box}
        Don't hide in the witness box
    \end{law}
    \begin{jury}[The Twelve]
        It could be you! So beware and see law~\ref{law:box}.
    \end{jury}
    \begin{jury}
        You will disregard the last statement.
    \end{jury}
    \begin{marg}No, No, No \end{marg}
    \begin{marg}Denis! \end{marg}
\end{minted}

示例输出:
\begin{quote}
    \theoremstyle{definition}   \newtheorem{law}{Law}
    \theoremstyle{plain}        \newtheorem{jury}[law]{Jury}
    \theoremstyle{remark}       \newtheorem*{marg}{Margaret}

    \begin{law}
        \label{law:box}
        Don't hide in the witness box
    \end{law}
    \begin{jury}[The Twelve]
        It could be you! So beware and see law~\ref{law:box}.
    \end{jury}
    \begin{jury}
        You will disregard the last statement.
    \end{jury}
    \begin{marg}No, No, No \end{marg}
    \begin{marg}Denis! \end{marg}
\end{quote}

在上面的示例中,“Jury”定理和“Law”定理用了相同的 \emph{counter} 参数,所以“Jury”的编号是接着“Law”的。

示例代码:
\begin{minted}{LaTeX}
    \newtheorem{mur}{Murphy}[section]

    \begin{mur}
        If there are two or more ways to do something, and one of those ways can
        result in a catastrophe, then someone will do it.
    \end{mur}
\end{minted}

示例输出:
\begin{quote}
    \newtheorem{mur}{Murphy}[section]

    \begin{mur}
        If there are two or more ways to do something, and one of those ways can
        result in a catastrophe, then someone will do it.
    \end{mur}
\end{quote}

上面示例中的“Murphy”定理由于定义时指定了 \emph{section} 参数,所以它的编号和当前章节有关。\emph{section} 参数
也可以换成 \emph{chapter} 或 \emph{subsection}。

\subsection{证明(Proofs)和证毕符号(End-of-Proof Symbol)}
\texttt{amsthm} 宏包也提供了 \texttt{proof} 环境。

示例代码:
\begin{minted}{LaTeX}
    \begin{proof}
        Trival, use
        \begin{equation*}
            E = mc^2.
        \end{equation*}
    \end{proof}

    \begin{proof}
        Trival, use
        \begin{equation*}
            E = mc^2. \qedhere
        \end{equation*}
    \end{proof}
\end{minted}

示例输出:
\begin{quote}
    \begin{proof}
        Trival, use
        \begin{equation*}
            E = mc^2.
        \end{equation*}
    \end{proof}

    \begin{proof}
        Trival, use
        \begin{equation*}
            E = mc^2. \qedhere
        \end{equation*}
    \end{proof}
\end{quote}

如果行末是一个不带编号的公式,证毕符号会另起一行,这时可以使用 \mintinline{LaTeX}{\qedhere} 命令将证毕符号放在
公式末尾。\mintinline{LaTeX}{\qedhere} 命令对 \texttt{IEEEeqnarray} 无效。这是由于 \texttt{IEEEeqnarray}
的内部结构导致的。为了保证公式可以水平居中,\texttt{IEEEeqnarray} 在公式块两边各加了一列。这两列只包含一个可以拉伸的
空格,是不可见的,因此 \mintinline{LaTeX}{\qedhere} 命令无法放在可拉伸空格的外面。

示例代码:
\begin{minted}{LaTeX}
    \begin{proof}
        This is a proof that ends with an equation array:
        \begin{IEEEeqnarray*}{rCl}
            a & = & b + c \\
            & = & d + e. \qedhere
        \end{IEEEeqnarray*}
    \end{proof}
\end{minted}

示例输出:
\begin{quote}
    \begin{proof}
        This is a proof that ends with an equation array:
        \begin{IEEEeqnarray*}{rCl}
            a & = & b + c \\
            & = & d + e. \qedhere
        \end{IEEEeqnarray*}
    \end{proof}
\end{quote}

对于上面的问题,有一种简单的解决方法,即对可拉伸的空格进行明确指定。

示例代码:
\begin{minted}{LaTeX}
    \begin{proof}
        This is a proof that ends with an equation array:
        \begin{IEEEeqnarray*}{+rCl+x*}
            a & = & b + c \\
            & = & d + e. & \qedhere
        \begin{IEEEeqnarray*}
    \end{proof}
\end{minted}

示例输出:
\begin{quote}
    \begin{proof}
        This is a proof that ends with an equation array:
        \begin{IEEEeqnarray*}{+rCl+x*}
            a & = & b + c \\
            & = & d + e. & \qedhere
        \end{IEEEeqnarray*}
    \end{proof}
\end{quote}

\texttt{\{+rCl+x*\}} 中的 \texttt{+} 表示可拉伸的空格,在公式左右各一列。现在又在右边可拉伸的空格外加了一个空白列
\texttt{x},这一列只有最后一行的 \mintinline{LaTeX}{\qedhere} 命令需要。最后又指定一个 \texttt{*},表示没有
空格,防止 \texttt{IEEEeqnarray} 在最右边添加额外的 \texttt{+} 空格。

对于有编号的公式,也有类似的问题和解决方法。

示例代码:
\begin{minted}{LaTeX}
    % Wrong with equation
    \begin{proof}
        This is a proof that ends with a nubmered equation:
        \begin{equation}
            a = b + c.
        \end{equation}
    \end{proof}

    % Right with equation
    \begin{proof}
        This is a proof that ends with a nubmered equation:
        \begin{equation}
            a = b + c. \qedhere
        \end{equation}
    \end{proof}

    % Wrong with IEEEeqnarray
    \begin{proof}
        This is a proof that ends with a equation array:
        \begin{IEEEeqnarray}{rCl}
            a & = & b + c \\
            & = & d + e.
        \end{IEEEeqnarray}
    \end{proof}

    % Right with IEEEeqnarray
    \begin{proof}
        This is a proof that ends with an equation array:
        \begin{IEEEeqnarray}{+rCl+x*}
            a & = & b + c \\
            & = & d + e. \\
            &&& \qedhere \nonumber
        \end{IEEEeqnarray}
    \end{proof}
\end{minted}

示例输出:
\begin{quote}
    % Wrong with equation
    \begin{proof}
        This is a proof that ends with a nubmered equation:
        \begin{equation}
            a = b + c.
        \end{equation}
    \end{proof}

    % Right with equation
    \begin{proof}
        This is a proof that ends with a nubmered equation:
        \begin{equation}
            a = b + c. \qedhere
        \end{equation}
    \end{proof}

    % Wrong with IEEEeqnarray
    \begin{proof}
        This is a proof that ends with a equation array:
        \begin{IEEEeqnarray}{rCl}
            a & = & b + c \\
            & = & d + e.
        \end{IEEEeqnarray}
    \end{proof}

    % Right with IEEEeqnarray
    \begin{proof}
        This is a proof that ends with an equation array:
        \begin{IEEEeqnarray}{+rCl+x*}
            a & = & b + c \\
            & = & d + e. \\
            &&& \qedhere \nonumber
        \end{IEEEeqnarray}
    \end{proof}
\end{quote}

\end{document}
