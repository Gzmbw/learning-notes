%-*- coding: UTF-8 -*-
% TypesettingText.tex
%
\documentclass[UTF8]{ctexart}
\usepackage{geometry}
\geometry{a4paper, centering, scale=0.8}
\usepackage{minted}
\usepackage{textcomp}
\usepackage[gen]{eurosym}
\usepackage{array}

\title{\heiti Chapter 2 Typesetting Text}
\author{\kaishu Du Ang \\ \texttt{du2ang233@gmail.com} }
\date{\today}

\begin{document}
\maketitle

\tableofcontents

\newpage
\section{文章和语言的结构}
写文章最重要的一点是要把想法、信息、知识传达给读者,而好的文章结构能够帮助读者更好地查看、感受和理解我们想传达的东西。

\LaTeX 中最重要的文本单位是段(paragraph)。一段文字应该只包含一个思想或一种想法。写文章时什么时候分段?应该怎么分段
呢?如果要写一个新的想法了,那就另起一段,对应在源码中空一行);否则,可以用换行符(line breaking)来继续写原来的想
法,对应在源码中使用 \mintinline{LaTeX}{\\} 或 \mintinline{LaTeX}{\newline}。

很多人都低估了合理分段的重要性。在\LaTeX 中,很多人甚至都不知道什么是分段,有时自己已经新起了一段都不知道。按照我的理解,
一般情况下是不需要使用换行符的,在该分段的时候在代码里空一行另起一段就行了。但是在使用公式(equation)的时候需要考虑好
该使用什么,这时候很容易犯上述的错误。下面是说明该换行还是该另起一段的三个正确示例:
\begin{minted}{LaTeX}
    % Example 1
    \ldots when Einstein introduced his formula
    \begin{equation}
        e = m \cdot c^2 \; ,
    \end{equation}
    which is at the same time the most widely known
    and the least well understood physical formula.

    % Example 2
    \ldots from which follows Kirchhoff’s current law:
    \begin{equation}
        \sum_{k=1}^{n} I_k = 0 \; .
    \end{equation}

    Kirchhoff’s voltage law can be derived \ldots

    % Example 3
    \ldots which has several advantages.
    \begin{equation}
        I_D = I_F - I_R
    \end{equation}
    is the core of a very different transistor model. \ldots
\end{minted}

\section{断行和断页}
\subsection{合理分段}
\begin{itemize}
    \item \mintinline{LaTeX}{\\} 或 \mintinline{LaTeX}{\newline}:断行、不另起一段。
    \mintinline{LaTeX}{\\} 也在表格、公式等地方用于分行,而 \mintinline{LaTeX}{\newline} 只用于文本段落中。
    \item \mintinline{LaTeX}{\\*}:断行、不另起一段、不断页。
    \item \mintinline{LaTeX}{\newpage} 或 \mintinline{LaTeX}{\clearpage}:断页。二者有些细微的区别:一是在
    双栏排版中 \mintinline{LaTeX}{\newpage} 只起到另起一栏的作用;二是涉及到浮动体的排版上行为不同。
    \item \mintinline{LaTeX}{\linebreak[n]}、\mintinline{LaTeX}{\nolinebreak[n]}、
    \mintinline{LaTeX}{\pagebreak[n]}、\mintinline{LaTeX}{\nopagebreak[n]}:向 \LaTeX 建议哪些地方适合断
    行、断页,哪些地方不适合断行、断页。n是数字,代表适合/不适合的程度,取值范围为0到4,默认为4。数字越大代表程度越高。
\end{itemize}

一般\LaTeX 都会找到最适合断行的地方。但是有些时候——比如它不知道该如何用连字符分割单词的时候,它可能会让一行文字从段落
右边伸出一部分,然后报出 \texttt{overfull hbox} 的警告。这时使用 \mintinline{LaTeX}{\sloppy} 命令可以使单词
之间的间距增大来避免这个问题,但是这时会报出 \texttt{underfull hbox} 的警告。大多数情况下这样排版出来的都不太好看。
可以使用 \mintinline{LaTeX}{\fussy} 命令恢复 \LaTeX 的默认方式。

\subsection{连字符(Hyphenation)}
对于绝大部分单词,\LaTeX 都能够找到合适的断词位置,在断开的行尾加上连字符 \texttt{-}。如果一些单词没能自动断词,我们
可以在单词内手动使用 \mintinline{LaTeX}{\-} 命令指定断词的位置。另外,也可以使用
\mintinline{LaTeX}{\hyphenation{word list}} 命令来指定使用连字符的位置,例如
\mintinline{LaTeX}|\hyphenation{FORTRAN Hy-phen-a-tion}|,其中的 \texttt{word list} 是不区分大小写的。

\mintinline{LaTeX}|\mbox{text}| 或 \mintinline{LaTeX}|\fbox{text}|:会避免 \texttt{text}被连字符分开。
\mintinline{LaTeX}|\fbox| 比 \mintinline{LaTeX}|\mbox| 多了个可见的框。

\section{预定义好的字符串}
\begin{itemize}
    \item \mintinline{LaTeX}{\today}:\today(打印当天日期)
    \item \mintinline{LaTeX}{\TeX}:\TeX
    \item \mintinline{LaTeX}{\LaTeX}:\LaTeX
    \item \mintinline{LaTeX}{\LaTeXe}:\LaTeXe
\end{itemize}

\section{特殊符号}
\subsection{引号(Quotation Marks)}
\begin{itemize}
    \item 双引号:\mintinline{LaTeX}{``...text...''}
    \item 单引号:\mintinline{LaTeX}{`...text...'}
\end{itemize}

\subsection{短划线(Dashes)和连字符(Hyphens)}
在 \LaTeX 中有下面四种横杠:
\begin{itemize}
    \item \mintinline{LaTeX}{-}:-,连字符(hyphen),用于连接词语
    \item \mintinline{LaTeX}{--}:--,短破折号(en-dash),常用于连接数字表示起止范围
    \item \mintinline{LaTeX}{---}:---,长破折号(em-dash),常用于表示意思的转换
    \item \mintinline{LaTeX}{$-$}:$-$,减号(minus sign)
\end{itemize}

\subsection{波浪线(Tilde)}
\begin{itemize}
    \item \mintinline{LaTeX}{\~{}}:\~{}
    \item \mintinline{LaTeX}{$\sim$}:$\sim$
\end{itemize}

\subsection{斜杠(Slash)}
\begin{itemize}
    \item \mintinline{LaTeX}{read/write}:read/write(不允许用连字符拆分)
    \item \mintinline{LaTeX}{read\slash write}:read\slash write(允许用连字符拆分)
\end{itemize}

\subsection{度(Degree Symbol)}
\begin{itemize}
    \item \mintinline{LaTeX}{$30\,^{\circ}\mathrm{C}$}\footnote{这里的 \mintinline{LaTeX}{\,} 会输出空
    格}:$30\,^{\circ}\mathrm{C}$
    \item \mintinline{LaTeX}{30 \textcelsius{}}:30 \textcelsius{}
    \item \mintinline{LaTeX}{86 \textdegree{}F}:86 \textdegree{}F
\end{itemize}

\subsection{欧元符号}
要想使用欧元符号,需要先在导言区通过 \mintinline{LaTeX}{\usepackage{textcomp}} 命令导入宏包,然后再使用\\
\mintinline{LaTeX}{\texteuro} 命令输出欧元符号。如果所用的字体不包含欧元符号或者想用别的字体的欧元符号,可以通过
\mintinline{LaTeX}{\usepackage[official]{eurosym}} 导入 \texttt{eurosym} 宏包,然后用
 \mintinline{LaTeX}{\euro} 输出官方的欧元符号。用\texttt{gen}来替换\texttt{official}参数可以使用和当前字体匹
 配的欧元符号。
\begin{itemize}
    \item \mintinline{LaTeX}{\texteuro}:\texteuro
    \item \mintinline{LaTeX}{\euro}:\euro
\end{itemize}

\subsection{省略号(Ellipsis)}
\LaTeX 提供了命令 \mintinline{LaTeX}{\ldots} 来生成省略号,相对于直接输入三个点的方式更为合理。
\mintinline{LaTeX}{\ldots} 和 \mintinline{LaTeX}{\dots} 是两个等效的命令。
\begin{itemize}
    \item \mintinline{LaTeX}{Apples, bananas, ...}:Apples, bananas, ...
    \item \mintinline{LaTeX}{Apples, bananas, \ldots}:Apples, bananas, \ldots
    \item \mintinline{LaTeX}{Apples, bananas, \dots}:Apples, bananas, \dots
\end{itemize}

\subsection{连字(Ligatures)}
有些相邻的字母在排版时会连接起来,可以通过 \mintinline{LaTeX}{\mbox{}} 命令避免它们相连。
\begin{itemize}
    \item \mintinline{LaTeX}{ffshfilfluffia}:ffshfilfluffia(相连的情况)
    \item \mintinline{LaTeX}{f\mbox{}fshf\mbox{}ilf\mbox{}luf\mbox{}f\mbox{}ia}:
    f\mbox{}fshf\mbox{}ilf\mbox{}luf\mbox{}f\mbox{}ia(没有相连的情况)
\end{itemize}

\subsection{重音(Accents)符号和特殊符号}
示例代码如下:
\begin{minted}{LaTeX}
    \emph{\=a}  \emph{\'a}  \emph{\v a}  \emph{\` a}

    H\^otel, na\"\i ve, \'el\`eve, \\
    sm\o rrebr\o d, !'Se\~norita!, \\
    Sch\"onbrunner Schlo\ss{}
    Stra\ss e
\end{minted}

示例输出结果如下:
\begin{quote}
    \emph{\=a} \qquad \emph{\'a} \qquad \emph{\v a} \qquad \emph{\` a}

    H\^otel, na\"\i ve, \'el\`eve, \\
    sm\o rrebr\o d, !`Se\~norita!, \\
    Sch\"onbrunner Schlo\ss{}
    Stra\ss e
\end{quote}

重音符号和特殊符号命令列表:
\begin{minted}{text}
    \`o    \'o    \^o    \~o
    \=o    \.o    \"o    \c c
    \u o   \v o   \H o   \c o
    \d o   \b o   \t oo
    \oe    \OE    \ae    \AE
    \aa    \AA
    \o     \O     \l     \L
    \i     \j     !`     ?`
\end{minted}

重音符号和特殊符号输出结果列表:
\begin{quote}
    \`o  \qquad  \'o  \qquad  \^o  \qquad  \~o \\
    \=o  \qquad  \.o  \qquad  \"o  \qquad  \c c \\
    \u o \qquad  \v o \qquad  \H o \qquad  \c o \\
    \d o \qquad  \b o \qquad  \t oo \\
    \oe  \qquad  \OE  \qquad  \ae  \qquad  \AE \\
    \aa  \qquad  \AA \\
    \o   \qquad  \O   ~\qquad  \l   ~\qquad  \L \\
    \i   ~\qquad  \j   ~~\qquad  !`   ~\qquad  ?` \\
\end{quote}

\section{国际语言支持/中文排版支持}
\LaTeX 对其他很多语言提供了支持。\texttt{babel} 宏包可以用于对各种语言进行适配。其他语言暂时也用不到,这里就记录一
下如何让 \LaTeX 支持中文。

使用 \LaTeX 排版中文有两种方式,一种是使用 \texttt{xeCJK} 宏包,另一种是使用 \CTeX 宏包和文档类,推荐使用后者。
\CTeX 宏包和文档类是对 \texttt{CJK} 和 \texttt{xeCJK} 等宏包的进一步封装。文档类包括 \texttt{ctexart}、
\texttt{ctexrep}、\texttt{ctexbook},分别是对 \LaTeX 的三个标准文档类 \texttt{article}、\texttt{report}、
\texttt{book} 的封装,对 \LaTeX 的排版样式做了许多调整,以切合中文排版风格。最新版本的 \CTeX 宏包/文档类甚至支持自
动配置字体。

\subsection{\CTeX 的安装}
\CTeX 宏集依赖的宏包和宏集已被最常见的 \TeX 发行版 \TeX Live 和 MiK\TeX 所收录。如果本地安装的 \TeX Live 或
MiK\TeX 不是完全版本,就需要通过这两个发行版提供的宏包管理器来安装宏包。

\TeX Live 的宏包管理器是 \texttt{tlmgr}。在Linux系统上,一般需要 \texttt{sudo} 权限才能正确地执行
\texttt{tlmgr} 的功能。

直接使用 \mintinline{bash}{sudo tlmgr [arg]} 时,可能会提示找不到 \texttt{tlmgr} 或没有这个命令。乍一看,情况
比较尴尬:不加 \texttt{sudo} 没有权限,加了 \texttt{sudo} 反而找不到命令了。经过上网搜索,在一个帖子
\footnote{\emph{sudo does not find tlmgr},
https://tex.stackexchange.com/questions/203874/sudo-does-not-find-tlmgr} 里找到了解决方法。原来,
\texttt{sudo} 有一种内置的保护机制,只会使用安全的环境变量 \texttt{PATH}。如果 \TeX Live 的路径不在
\texttt{sudo} 的安全环境变量内,它就找不到相关的命令。可以在终端执行
\mintinline{bash}{sudo gedit /etc/sudoers},然后将\TeX Live的路径添加到 \texttt{sudo} 的
\texttt{secure\_path}中。在我的 Ubuntu 16.04 上,添加后结果如下(后面的原有路径省略,不同路径用 \texttt{:}
隔开):
\begin{minted}{text}
Defaults secure_path="/usr/local/texlive/2016/bin/x86_64-linux:/usr/local/sbin:..."
\end{minted}

不能用 \texttt{sudo} 执行 \texttt{tlmgr} 的问题解决后,在终端中依次执行以下命令,以更新 \texttt{tlmgr} 宏包管
理器、已安装的所有宏包、安装 \CTeX 宏集。
\begin{minted}{bash}
    sudo tlmgr update --self
    sudo tlmgr update --all
    sudo tlmgr install ctex
\end{minted}

\subsection{使用 \CTeX 文档类}
\CTeX 宏集提供了四个中文文档类:\texttt{ctexart}、\texttt{ctexrep}、\texttt{ctexbook} 和
\texttt{ctexbeamer},分别对应 \LaTeX 的标准文档类 \texttt{article}、\texttt{report}、\texttt{book} 和
\texttt{beamer}。使用它们的时候,需要将涉及到的所有源文件使用 UTF-8 编码保存。

下面是使用 \texttt{ctexart} 文档类编写的一个例子:
\begin{minted}{LaTeX}
    \documentclass[UTF8]{ctexart}
    \begin{document}
    中文文档类测试。你需要将所有源文件保存为 UTF-8 编码。
    你可以使用 XeLaTeX、LuaLaTeX 或 upLaTeX 编译,也可以使用 (pdf)LaTeX 编译。
    推荐使用 XeLaTeX 或 LuaLaTeX 编译。
    \end{document}
\end{minted}

\CTeX 预定义的字库中的中文字体已经基本够用,包括{\songti 宋体}(\mintinline{LaTeX}{\songti})、{\heiti 黑体}
(\mintinline{LaTeX}{\heiti})、{\kaishu 楷书}(\mintinline{LaTeX}{\kaishu})、{\fangsong 仿宋}
(\mintinline{LaTeX}{\fangsong})等。更多 \CTeX 的使用参考《\CTeX 宏集手册》
\footnote{《\CTeX 宏集手册》,http://mirror.unl.edu/ctan/language/chinese/ctex/ctex.pdf}。

\section{单词之间的空格}
为了使输出更美观、更具可读性,\LaTeX 可能会在不同单词之间或句子末尾插入更多空格。\LaTeX 默认句子以句点(periods)、
问号(question marks)或者感叹号(exclamation marks)结尾。但是如果句点跟在一个大写字母后面,它不会认为这是句子结
尾,因为大写字母后面跟句点往往是缩略词。

用户可以通过具体的命令来改变上面的默认设定。一个斜杠跟一个空格会产生一个不会被扩大的空格;一个波浪线(\texttt{$\sim$})
会产生一个既不能被扩大、也不能从这里断行的空格;在句点前使用 \mintinline{LaTeX}{\@} 命令,不管这个句点是不是跟在大
写字母后面,都会指定这个句子到句点就结束。使用 \mintinline{LaTeX}{\frenchspacing} 命令可以强制不在一个句子后面插
入多余的空格。如果使用 \mintinline{LaTeX}{\frenchspacing} 命令就没必要再用 \mintinline{LaTeX}{\@} 了。


代码示例:
\begin{minted}{LaTeX}
    Mr.~Smith was happy to see her\\
    cf.~Fig.~5\\
    I like BASIC\@. What about you?
\end{minted}

示例输出:
\begin{quote}
    Mr.~Smith was happy to see her\\
    cf.~Fig.~5\\
    I like BASIC\@. What about you?
\end{quote}

\section{标题、章、节}
文档类 \texttt{article} 中有以下几种分层次结构的命令:
\begin{minted}{LaTeX}
    \section{...}
    \subsection{...}
    \subsubsection{...}
    \paragraph{...}
    \subparagraph{...}
\end{minted}

\mintinline{LaTeX}{\part{...}} 命令也可以把文档分为多个部分,但它不会影响 \texttt{section} 和
\texttt{chapter} 的编号。

和 \texttt{article} 文档类相比,在 \texttt{report} 和 \texttt{book} 中, 可以使用
\mintinline{LaTeX}{\chapter{...}}。

由于 \texttt{article} 文档类中不包含 \texttt{chapter},所以可以很方便地把 \texttt{article} 作为
\texttt{chapter} 插入 \texttt{book} 文档类。章节空隙、编号等由 \LaTeX 自动完成。

下面是两个比较特殊的情况:
\begin{itemize}
    \item \mintinline{LaTeX}{\part} 命令不会影响 \texttt{chapter} 或 \texttt{section} 的编号
    \item \mintinline{LaTeX}{\appendix} 命令没有任何参数,会把 \texttt{chapter} (对于 \texttt{report}、
    \texttt{book})或 \texttt{section} (对于 \texttt{article})的数字编号转换成字母编号。
\end{itemize}

\mintinline{LaTeX}{\tableofcontents} 命令可以用于建立目录,目录就会在这条命令所在的位置生成。一般新写的文档需要
编译两次才能正确生成目录,必要的时候 \LaTeX 也会提示需要编译三次。

上面提到的分章节的命令都有一个加星号的版本,就是在原来的命令名称后面加一个星号,成为稍有不同的新命令。例如
\mintinline{LaTeX}{\section{Help}} 命令,加星号之后的命令为 \mintinline{LaTeX}{\section*{Help}}。加星版本
的章节命令对应的标题不会显示在目录里,也不会被编号。

有时候章节的标题太长,这会导致其在目录里显示不佳。可以通过下面的命令在真正的标题前选择添加一个参数,指定在目录中显示的标
题。
\begin{minted}{LaTeX}
    \chapter[Title for the table of contents]{A long
        and especially boring title, shown in the text}
\end{minted}

整个文档的标题是通过 \mintinline{LaTeX}{\maketitle} 命令产生的。在调用 \mintinline{LaTeX}{\maketitle} 命令
之前,文档标题的内容需要由 \mintinline{LaTeX}{\title{...}}、\mintinline{LaTeX}{\author{...}}、
\mintinline{LaTeX}{\date{...}}(可选)等参数指定。在 \mintinline{LaTeX}{\author{...}} 命令的参数中,可以用
\mintinline{LaTeX}{\and} 来间隔多个作者名字。

所有标准文档类都提供了一个 \mintinline{LaTeX}{\appendix} 命令将正文和附录分开,使用
\mintinline{LaTeX}{\appendix} 后,最高一级章节改为使用拉丁字母编号,从A开始。

\LaTeXe 在 \texttt{book} 文档类有以下三个额外的命令,可以进行前言、正文、后记的结构划分。这三个命令还可和
\mintinline{LaTeX}{\appendix} 命令结合,生成有前言、正文、附录、后记四部分的文档。
\begin{itemize}
    \item \mintinline{LaTeX}{\frontmatter} 前言部分,放置在文档主体的最开始
    (\mintinline{LaTeX}{\begin{document}}),它会把页码变成罗马数字,其后的 \mintinline{LaTeX}{\chapter}
    不编号
    \item \mintinline{LaTeX}{\mainmatter} 正文部分,页码为阿拉伯数字格式,从 1 开始计数,其后的章节编号正常
    \item \mintinline{LaTeX}{\backmatter} 后记部分,页码格式不变,继续正常计数;其后的
    \mintinline{LaTeX}{\chapter} 不编号
\end{itemize}

\section{交叉引用(Cross References)}
在写作时经常要用到对图片、表格等的交叉引用,在 \LaTeX 中可以用下面的命令完成:
\begin{minted}{LaTeX}
\label{marker}, \ref{marker}, \pageref{marker}
\end{minted}
其中,\texttt{marker} 是由用户自行定义的标识符。
示例代码:
\begin{minted}{LaTeX}
A reference to this subsection \label{sec:this} looks like:
``see section~\ref{sec:this} on page~\pageref{sec:this}.''
\end{minted}
示例输出:
A reference to this subsection \label{sec:this} looks like: ``see section~\ref{sec:this} on
page~\pageref{sec:this}.''

\section{脚注(Footnotes)}
可以使用 \mintinline{LaTeX}{\footnote{...}} 命令来添加脚注,脚注应该紧跟在它注解的词或句子(包括标点符号)后面。

由于脚注会分散读者的注意力,所以尽量在文章主体说清楚,少用脚注。

\section{强调(Emphasis)}
以前用打印机打字的时候,习惯把重要的词用 \underline{下划线} 来强调,这在 \LaTeX 中可以通过
\mintinline{LaTeX}{\underline{...}} 命令来实现。但在印刷书籍中,一般通过 \mintinline{LaTeX}{\emph{...}} 命
令,使用意大利字体(\emph{italic} font)进行强调。

\mintinline{LaTeX}{\emph{...}} 命令使用意大利字体进行强调并不是绝对的,这还要结合具体的语境。

\newpage
示例代码:
\begin{minted}{LaTeX}
    \emph{If you use emphasizing inside a piece of emphasized text, then \LaTeX{} uses the
        \emph{normal} font for emphasizing.}
\end{minted}

示例输出:
\begin{quote}
    \emph{If you use emphasizing inside a piece of emphasized text, then \LaTeX{} uses the
        \emph{normal} font for emphasizing.}
\end{quote}

\section{环境(Environments)}
环境的典型命令为 \mintinline{LaTeX}{\begin{\emph{environment}} text \end{\emph{environment}}},其中
\emph{environment} 是环境的名字。

环境可以相互嵌套,例如:
\begin{minted}{LaTeX}
    \begin{aaa}
        ...
        \begin{bbb}
            ...
        \end{bbb}
        ...
    \end{aaa}
\end{minted}
\subsection{Itemize,Enumerate 和 Description}
示例代码:
\begin{minted}{LaTeX}
    \flushleft % 左对齐
    \begin{enumerate}
        \item You can nest the list environments to your taste:
        \begin{itemize}
            \item But it might start to look silly.
            \item[-] With a dash.
        \end{itemize}
        \item Therefore remember:
        \begin{description}
            \item[Stupid] things will not become smart because they are in a list.
            \item[Smart] things, though, can be presented beautifully in a list.
        \end{description}
    \end{enumerate}
\end{minted}
\newpage
示例输出:
\begin{quote}
    \flushleft
    \begin{enumerate}
        \item You can nest the list environments to your taste:
        \begin{itemize}
            \item But it might start to look silly.
            \item[-] With a dash.
        \end{itemize}
        \item Therefore remember:
        \begin{description}
            \item[Stupid] things will not become smart because they are in a list.
            \item[Smart] things, though, can be presented beautifully in a list.
        \end{description}
    \end{enumerate}
\end{quote}

\subsection{Flushleft,Flushright 和 Center}
\texttt{flushleft}、\texttt{flushright} 和 \texttt{center} 环境分别会使段落左对齐、右对齐和居中。

示例代码:
\begin{minted}{LaTeX}
    \begin{flushleft}
        This text is\\ left-aligned.
        \LaTeX{} is not trying to make
            each line the same length.
    \end{flushleft}

    \begin{flushright}
        This text is right-\\aligned.
        \LaTeX{} is not trying to make
            each line the same length.
    \end{flushright}

    \begin{center}
        At the centre\\of the earth
    \end{center}
\end{minted}

示例输出:
\begin{quote}
    \begin{flushleft}
        This text is\\ left-aligned.
        \LaTeX{} is not trying to make
            each line the same length.
    \end{flushleft}

    \begin{flushright}
        This text is right-\\aligned.
        \LaTeX{} is not trying to make
            each line the same length.
    \end{flushright}

    \begin{center}
        At the centre\\of the earth
    \end{center}
\end{quote}

\subsection{Quote,Quotation 和 Verse}
\texttt{quote} 环境适合引用一些名言、重要的词句、示例等。
\newpage
示例代码:
\begin{minted}{LaTeX}
    A typographical rule of thumb for the line length is:
    \begin{quote}
        On average, no line should be longer than 66 characters.
    \end{quote}
    This is why \LaTeX{} pages have such large borders by default and also why
         multicolumn print is used in newspapers.
\end{minted}

示例输出:
\begin{quote}
    A typographical rule of thumb for the line length is:
    \begin{quote}
        On average, no line should be longer than 66 characters.
    \end{quote}
    This is why \LaTeX{} pages have such large borders by default and also why
         multicolumn print is used in newspapers.
\end{quote}

\texttt{quotation} 环境和 \texttt{verse} 环境很像。但由于 \texttt{quotation} 环境对每一段都会缩进,所以适合
引用比较长的、一般有几段的内容;而 \texttt{verse} 环境很适合引用诗歌,利用 \mintinline{LaTeX}{\\} 或空行来分行。

示例代码:
\begin{minted}{LaTeX}
    I know only one English poem by heart. It is about Humpty Dumpty.
    \begin{flushleft}
        \begin{verse}
            Humpty Dumpty sat on a wall:\\
            Humpty Dumpty had a great fall.\\
            All the King’s horses and all
            the King’s men\\
            Couldn’t put Humpty together
            again.
        \end{verse}
    \end{flushleft}
\end{minted}

示例输出:
\begin{quote}
    I know only one English poem by heart. It is about Humpty Dumpty.
    \begin{flushleft}
        \begin{verse}
            Humpty Dumpty sat on a wall:\\
            Humpty Dumpty had a great fall.\\
            All the King’s horses and all
            the King’s men\\
            Couldn’t put Humpty together
            again.
        \end{verse}
    \end{flushleft}
\end{quote}

\subsection{Abstract}
在科学刊物中,一般会以一段摘要开头,让读者对这篇文章有一个整体的认知,这是一个惯例。\LaTeX 的 \texttt{abstract} 环
境就是用于写摘要的,一般用于 \texttt{article} 文档类。
\newpage
示例代码:
\begin{minted}{LaTeX}
    \begin{abstract}
        The abstract contents.
    \end{abstract}
\end{minted}

\subsection{Verbatim}
在 \mintinline{LaTeX}{verbatim} 和 \mintinline{LaTeX}{verbatim} 之间的所有文本都会原封不动地打印出来,文本中的
\LaTeX 命令不会被执行,所以可以用 \texttt{verbatim} 环境来插入一段代码。

如果要插入行间代码,可以用 \mintinline{LaTeX}{\verb+text+} 命令。其中的 \texttt{+} 只是分隔符,可以由用户随意指
定,但不能用字母、\texttt{*} 或空格,习惯上用 \texttt{|}。

示例代码:
\begin{minted}{LaTeX}
    The \verb|\ldots| command \ldots

    \begin{verbatim}
        10 PRINT "HELLO WORLD ";
        20 GOTO 10
    \end{verbatim}
\end{minted}

示例输出:
\begin{quote}
The \verb|\ldots| command \ldots

\begin{verbatim}
    10 PRINT "HELLO WORLD ";
    20 GOTO 10
\end{verbatim}
\end{quote}

\texttt{verbatim} 环境和 \texttt{verb} 命令都各自有一个加 \texttt{*} 的版本,区别是会把空格显示成 \verb*| |。

示例代码:
\begin{minted}{LaTeX}
    \verb*|like   this :-) |

    \begin{verbatim*}
    the starred version of the verbatim environment emphasizes the spaces
    in the text.
    \end{verbatim*}
\end{minted}

示例输出:
\begin{quote}
\verb*|like   this :-) |

\begin{verbatim*}
the starred version of the verbatim environment emphasizes the spaces
in the text.
\end{verbatim*}
\end{quote}

\texttt{verbatim} 环境和 \verb|\verb| 命令对符号的处理比较复杂,一般不能用在其它命令的参数里,否则多半会出错。

\texttt{verbatim} 宏包优化了 \texttt{verbatim} 环境的内部命令,并提供了
\mintinline{LaTeX}|\verbatiminput| 命令用来直接读入文件生成代码环境。\texttt{fancyvrb} 宏包提供了可定制格式的
\texttt{verbatim} 环境;\texttt{listings} 宏包更进一步,可生成关键字高亮的代码环境,支持各种程序设计语言的语法和
关键字;本篇笔记使用的是 \texttt{minted} 宏包,代码可以有彩色高亮。详情请参考各自的帮助手册。

\subsection{Tabular}
\texttt{tabular} 环境可以用于排版表格,\LaTeX 会自动地调整表格每一列的宽度。具体命令如下:
\mint{LaTeX}{   \begin{tabular}[pos]{talbe spec}}
其中,\texttt{table spec} 参数决定了表格的格式,\texttt{l}、\texttt{r}、\texttt{c} 分别会使单元格内容左对齐、
右对齐、居中,不折行;\texttt{|} 会绘制竖线;\texttt{p\{\emph{width}\}} 会使单元格固定宽度为
\texttt{\emph{width}},可以自动折行;可以通过 \texttt{@\{...\}} 来在单元格前后插入任意的文本,但同时它会消除单元
格前后额外添加的间距。

表格中每行的单元格数目不能多于列格式里 l/c/r/p 的总数(可以少于这个总数),否则出错。

\texttt{pos} 参数用于指定表格相对于环绕为基线的垂直位置,可以为 \texttt{t}、\texttt{b} 和 \texttt{c},分别代表
顶部(top)、底部(bottom)和中间(center)。

在 \texttt{tabular} 环境内,\texttt{\&} 会跳到下一列,\mintinline{LaTeX}{\\} 会新起一行,
\mintinline{LaTeX}{\hline} 会插入一条水平线。使用 \mintinline{LaTeX}{\cline{i-j}} 会插入一部分水平线,其中
\texttt{i} 和 \texttt{j} 是列的编号。

示例代码 1:
\begin{minted}{LaTeX}
    \begin{tabular}{|r|l|}
    \hline
    7C0 & hexadecimal \\
    3700 & octal \\ \cline{2-2}
    11111000000 & binary \\
    \hline \hline
    1984 & decimal \\
    \hline
    \end{tabular}
\end{minted}

示例输出 1:
\begin{quote}
    \begin{tabular}{|r|l|}
    \hline
    7C0 & hexadecimal \\
    3700 & octal \\ \cline{2-2}
    11111000000 & binary \\
    \hline \hline
    1984 & decimal \\
    \hline
    \end{tabular}
\end{quote}

示例代码 2:
\begin{minted}{LaTeX}
    \begin{tabular}{|p{4.7cm}|}
    \hline
    Welcome to Boxy's paragraph.
    We Sincerely hope you'll all enjoy the show.\\
    \hline
    \end{tabular}
\end{minted}

示例输出 2:
\begin{quote}
    \begin{tabular}{|p{4.7cm}|}
    \hline
    Welcome to Boxy's paragraph.
    We Sincerely hope you'll all enjoy the show.\\
    \hline
    \end{tabular}
\end{quote}

如果想让表格的某一列按小数点对齐,又不使用额外的宏包(\texttt{dcolumn}),可以采用一种比较折中的方法:把该列的所有小
数分为整数部分和小数部分,各作为一列,然后在 \texttt{tabular} 环境中使用 \texttt{@{.}} 作为整数列和小数列的分隔
符。整数部分和小数部分之间不要忘了用 \texttt{\&} 隔开,因为它们现在是两列了。为了在这两列的表头仅显示一个标签,可以使用
\mintinline{LaTeX}{\multicolumn} 命令。

示例代码 3:
\begin{minted}{LaTeX}
    \begin{tabular}{c r @{.} l}
    Pi expression       &
    \multicolumn{2}{c}{Value} \\
    \hline
    $\pi$               & 3&1416  \\
    $\pi^{\pi}$         & 36&46   \\
    $(\pi^{\pi})^{\pi}$ & 80662&7 \\
    \end{tabular}
    \qquad
    \begin{tabular}{|c|c|}
    \hline
    \multicolumn{2}{|c|}{Ene} \\
    \hline
    Mene & Muh! \\
    \hline
    \end{tabular}
\end{minted}

示例输出 3:
\begin{quote}
    \begin{tabular}{c r @{.} l}
    Pi expression       &
    \multicolumn{2}{c}{Value} \\
    \hline
    $\pi$               & 3&1416  \\
    $\pi^{\pi}$         & 36&46   \\
    $(\pi^{\pi})^{\pi}$ & 80662&7 \\
    \end{tabular}
    \qquad
    \begin{tabular}{|c|c|}
    \hline
    \multicolumn{2}{|c|}{Ene} \\
    \hline
    Mene & Muh! \\
    \hline
    \end{tabular}
\end{quote}

有时需要为整列修饰格式,比如整列改变为粗体,如果每个单元格都加上 \mintinline{LaTeX}{\bfseries} 命令会比较麻烦。
\texttt{array} 宏包提供了辅助格式 \texttt{>} 和 \texttt{<},用于给列格式前后加上修饰命令。辅助格式甚至支持插入
\mintinline{LaTeX}{\centering} 等命令改变\texttt{p\{\emph{width}\}} 列格式的对齐方式,一般还要加额外的命令
\mintinline{LaTeX}{\arraybackslash} 以免出错。因为 \mintinline{LaTeX}{\centering} 等对齐命令会破坏表格环境
里 \mintinline{LaTeX}{\\} 换行命令的定义,\mintinline{LaTeX}{\arraybackslash} 用来恢复之。如果不加
\mintinline{LaTeX}{\arraybackslash} 命令,也可以用 \mintinline{LaTeX}{\tabularnewline} 命令代替原来的
\mintinline{LaTeX}{\\} 实现表格换行。
\newpage
示例代码 4:
\begin{minted}{LaTeX}
    \begin{tabular}{>{\itshape}r<{*}l}
    \hline
    italic & normal \\
    column & column \\
    \hline
    \end{tabular}

    \begin{tabular}
    {>{\centering\arraybackslash}p{9em}}
    \hline
    Some center-aligned long text. \\
    \hline
    \end{tabular}
\end{minted}

示例输出 4:
\begin{quote}
    \begin{tabular}{>{\itshape}r<{*}l}
    \hline
    italic & normal \\
    column & column \\
    \hline
    \end{tabular}

    \begin{tabular}
    {>{\centering\arraybackslash}p{9em}}
    \hline
    Some center-aligned long text. \\
    \hline
    \end{tabular}
\end{quote}

有时 \LaTeX 默认的表格会比较拥挤,可以通过改变 \verb|\arraystretch| 和 \verb|\tabcolsep| 参数来调节。

示例代码 5:
\begin{minted}{LaTeX}
    \begin{tabular}{|l|}
    \hline
    These lines\\\hline
    are tight\\\hline
    \end{tabular}

    {\renewcommand{\arraystretch}{1.5}
    \renewcommand{\tabcolsep}{0.2cm}
    \begin{tabular}{|l|}
    \hline
    less cramped\\\hline
    table layout\\\hline
    \end{tabular}}
\end{minted}

示例输出 5:
\begin{quote}
    \begin{tabular}{|l|}
    \hline
    These lines\\\hline
    are tight\\\hline
    \end{tabular}

    {\renewcommand{\arraystretch}{1.5}
    \renewcommand{\tabcolsep}{0.2cm}
    \begin{tabular}{|l|}
    \hline
    less cramped\\\hline
    table layout\\\hline
    \end{tabular}}
\end{quote}

如果只想增加表格中某一行的高度,可以通过添加一条不可见的“支柱”(\texttt{bar})来实现。即使用
\mintinline{LaTeX}{\rule} 命令,并把宽度设置为0。

示例代码 6:
\begin{minted}{LaTeX}
    \begin{tabular}{|c|}
    \hline
    \rule{1pt}{4ex}Pitprop \ldots\\ % 宽度不为0,可见
    \hline
    \rule{0pt}{4ex}Strut\\          % 宽度为0,不可见
    \hline
    \end{tabular}
\end{minted}

示例输出 6:
\begin{quote}
    \begin{tabular}{|c|}
    \hline
    \rule{1pt}{4ex}Pitprop \ldots\\ % 宽度不为0,可见
    \hline
    \rule{0pt}{4ex}Strut\\          % 宽度为0,不可见
    \hline
    \end{tabular}
\end{quote}

如果想排版较长的表格,可以使用 \texttt{longtable} 环境。在 \texttt{booktabs} 宏包中,还有很多其他的命令可以美化
表格。

\section{浮动体(Floating Bodies)}
一般的刊物、出版品都会有各种图表,由于这类元素不能跨页,所以需要特殊对待。有一种方法是每当遇到一个图或表在当前页放不开
的时候,就新起一页。这种方法会让页面有比较多的留白,看起来会不美观。

一种比较好的解决方法是当前页放不开这个图表时,把它们“浮动”到后面的页面去,当前页面用正文进行填充。\LaTeX 提供了两种浮
动体环境:\texttt{figure} 和 \texttt{table},分别针对图片和表格。任何放在 \texttt{figure} 和 \texttt{table}
环境中的东西都会被视为是浮动体。
\mint{LaTeX}{   \begin{figure}[placement specifier] or \begin{table}[...]}
所有的浮动体环境都会有一个叫 \texttt{placement specifier} 的可选的参数,可以用于指定这个浮动体被允许移动到哪里。下
面是 \texttt{placement specifier} 参数的可选值(默认为 \texttt{tbp},优先级按照 \texttt{h-t-b-p} 排列,与顺
序无关):
\begin{itemize}
    \item \texttt{h},当前位置(代码所处的上下文)
    \item \texttt{t},顶部,如果是当页排版可能出现在代码之前
    \item \texttt{b},底部
    \item \texttt{p},一个或多个浮动体被放在单独的页面中,这个页面被称为浮动页(float page),与之对应,有文本的页
    面称为文本页(text page)
    \item \texttt{!},在决定位置时忽略文本页的限制(\LaTeX 对每个位置的浮动体的总数和占用大小等有一定的限制),只
    有浮动页的限制(\mintinline{LaTeX}{\floatpagefraction} 和
    \mintinline{LaTeX}{\dblfloatpagefraction})起效。
\end{itemize}

\LaTeX 会根据用户指定的 \texttt{placement specifier} 参数来放置每个浮动体。如果一个浮动体不能被放置在当前页,它
就会被加入 \texttt{figures} 队列或 \texttt{tables} 队列。当新起一页的时候,\LaTeX 会先检查能否把这一页作为一个
浮动页来放置队列中的浮动体。如果不行,队列中的第一个浮动体就会被认为是这个新起的一页刚产生的,然后 \LaTeX 就再次根据它
们的 \texttt{placement specifier} 参数在这个新起的一页进行放置。\LaTeX 会严格地按照浮动体在队列中的顺序来执行这
一过程,这也就是为什么在如果 \texttt{figures} 队列中的第一个图片一直无法放置,那么所有的图片都会被拖到文档最后的原
因。因此,如果 \LaTeX 如果没有按照要求放置浮动体,在两个浮动体队列中经常会出现一个干扰下一个的情况。

尽管可以只指定一个值给 \texttt{placement specifier} 参数,但这样往往会产生问题,一个浮动体没法放置,之后的浮动体
也会受到阻碍无法放置。永远都不要只给 \texttt{placement specifier} 参数设置一个 \texttt{h},这很容易产生问题,以
至于新版本的 \LaTeX 会自动将其替换为 \texttt{ht}。

\texttt{float} 宏包为浮动体提供了 \texttt{H} 位置参数,不与 \texttt{htbp} 及 \texttt{!} 混用。使用
\texttt{H} 位置参数时,会取消浮动机制,将浮动体视为一般的盒子插入当前位置。

用 \mintinline{LaTeX}{\caption{caption text}} 命令来给 \texttt{table} 或 \texttt{figure} 环境加说明文
字,\LaTeX 会自动根据其类型分别编号。

\mintinline{LaTeX}{\listoffigures} 和 \mintinline{LaTeX}{\listoftables} 命令和
\mintinline{LaTeX}{\tableofcontents} 命令很类似,可以用于建立图表索引。这种图表索引会完整地显示所有图标的说明文
字,所以如果说明文字太长也不太好,这时可以用 \mintinline{LaTeX}{\caption[Short][LLLLLLLooooonnnnnggggggg]}
命令来指定一个对应的短点的说明。

可以通过 \mintinline{LaTeX}{\label} 和 \mintinline{LaTeX}{\ref} 命令来对浮动体建立标签和进行引用。需要注意,
既然是要引用图表的编号,所以要把 \mintinline{LaTeX}{\label} 命令放置在 \mintinline{LaTeX}{\caption}之后。

示例代码:
\begin{minted}{LaTeX}
    Figure~\ref{white} is an example of Pop-Art.
    \begin{figure}[!hbtp]
    \makebox[\textwidth]{\framebox[5cm]{\rule{0pt}{5cm}}}
    \caption{Five by Five in Centimetres.\label{white}}
    \end{figure}
\end{minted}

在上里面的示例中,\texttt{[!hbtp]} 是 \LaTeX 中最宽松的浮动体位置参数。\LaTeX 会先努力尝试(\texttt{!})将图片
放置在产生它代码所在的上下文位置(\texttt{h})。如果不行,它会尝试放置在该页的底部(\texttt{b})。如果还是不行,就不
能放在该页了,它会考虑能不能建立一个浮动体页来放置这个浮动体以及其他的浮动体。如果不满足新建一个浮动体页的条件,那
\LaTeX 就会新起一页,并把它看作是新起的一页刚产生的浮动体,并按照它的参数重新放置。


在遇到 \mintinline{LaTeX}{\clearpage}、\mintinline{LaTeX}{\cleardoublepage} 或
\mintinline{LaTeX}{\end{document}} 这三个命令时,所有队列中未处理的浮动体都会直接输出,
\texttt{placement specifier} 参数中的 \texttt{p} 选项也会打开以保证可以将所有的浮动体输出。

\section{保护脆弱命令(Protecting Fragile Commads)}
像 \mintinline{LaTeX}{\caption} 和 \mintinline{LaTeX}{\section} 这种命令以文本作为参数,而且这些文本参数在
文档中可能不只出现一次,所以这些文本参数称为移动参数~\footnote{\emph{Moving Arguments And Fragile Commands},
http://www.dickimaw-books.com/latex/novices/html/fragile.html}。但有些脆弱命令,如
\mintinline{LaTeX}{\footnote} 和 \mintinline{LaTeX}{\phantom} 等,作为它们的参数的时候会失效,甚至引起报错。
需要在脆弱命令前加上 \mintinline{LaTeX}{\protect} 命令来保护它们,这样它们就可以作为移动参数了。

\mintinline{LaTeX}{\protect} 命令只会影响紧跟在它之后的一个命令,一般多加了 \mintinline{LaTeX}{\protect} 命
令也不会出什么问题。

\end{document}
