%-*- coding: UTF-8 -*-
% TypesettingText.tex
%
\documentclass[UTF8]{ctexart}
\usepackage{geometry}
\geometry{a4paper, centering, scale=0.8}
\usepackage{minted}
\usepackage{textcomp}
\usepackage[gen]{eurosym}

\title{\heiti Chapter 2 Typesetting Text}
\author{\kaishu Du Ang \\ \texttt{du2ang233@gmail.com} }
\date{\today}

\begin{document}
\maketitle

\section{文章和语言的结构}
写文章最重要的一点是要把想法、信息、知识传达给读者,而好的文章结构能够帮助读者更好地查看、感受和理解我们想传达的东西。

\LaTeX 中最重要的文本单位是段(paragraph)。一段文字应该只包含一个思想或一种想法。写文章时什么时候分段?应该怎么分段呢?如果要写一个新的想法了,
那就另起一段,对应在源码中空一行);否则,可以用换行符(line breaking)来继续写原来的想法,对应在源码中使用
\mintinline{LaTeX}{\\} 或 \mintinline{LaTeX}{\newline}。

很多人都低估了合理分段的重要性。在\LaTeX 中,很多人甚至都不知道什么是分段,自己已经新起了一段都不知道。按照我的理解,一般情况下是不需要使用换
行符的,在该分段的时候在代码里空一行另起一段就行了。但是在使用公式(equation)的时候需要考虑好该使用什么,这时候很容易犯上述的错误。下面是说明
该换行还是该另起一段的三个正确示例:
\begin{minted}{LaTeX}
    % Example 1
    \ldots when Einstein introduced his formula
    \begin{equation}
        e = m \cdot c^2 \; ,
    \end{equation}
    which is at the same time the most widely known
    and the least well understood physical formula.

    % Example 2
    \ldots from which follows Kirchhoff’s current law:
    \begin{equation}
        \sum_{k=1}^{n} I_k = 0 \; .
    \end{equation}

    Kirchhoff’s voltage law can be derived \ldots

    % Example 3
    \ldots which has several advantages.
    \begin{equation}
        I_D = I_F - I_R
    \end{equation}
    is the core of a very different transistor model. \ldots
\end{minted}

\section{断行和断页}
\subsection{合理分段}
\begin{itemize}
    \item \mintinline{LaTeX}{\\} 或 \mintinline{LaTeX}{\newline}:断行、不另起一段。
    \mintinline{LaTeX}{\\} 也在表格、公式等地方用于分行,而 \mintinline{LaTeX}{\newline} 只用于文本段落中。
    \item \mintinline{LaTeX}{\\*}:断行、不另起一段、不断页。
    \item \mintinline{LaTeX}{\newpage} 或 \mintinline{LaTeX}{\clearpage}:断页。二者有些细微的区别:一是在双栏排版
    中 \mintinline{LaTeX}{\newpage} 只起到另起一栏的作用;二是涉及到浮动体的排版上行为不同。
    \item \mintinline{LaTeX}{\linebreak[n]}、\mintinline{LaTeX}{\nolinebreak[n]}、\mintinline{LaTeX}{\pagebreak[n]}、
    \mintinline{LaTeX}{\nopagebreak[n]}:向 \LaTeX 建议哪些地方适合断行、断页,
    哪些地方不适合断行、断页。n是数字,代表适合/不适合的程度,取值范围为0到4,默认为4。数字越大代表程度越高。
\end{itemize}

一般\LaTeX 都会努力找到最适合断行的地方。但是有些时候——比如它不知道该如何用连字符分割单词的时候,它可能会让这一行文字从右边伸出这一段,然后
报出\texttt{overfull hbox}的警告。这时使用 \mintinline{LaTeX}{\sloppy} 命令可以使单词之间的间距增大来避免这个问题,但是这时会报出
\texttt{underfull hbox}的警告。大多数情况下这样排版出来的都不太好看。可以使用 \mintinline{LaTeX}{\fussy} 命令恢复 \LaTeX 的默认方式。

\section{连字符(Hyphenation)}
对于绝大部分单词,\LaTeX 都能够找到合适的断词位置,在断开的行尾加上连字符 \texttt{-}。如果一些单词没能自动断词,我们可以在单词内手动使用
\mintinline{LaTeX}{\-} 命令指定断词的位置。另外,也可以使用 \mintinline{LaTeX}{\hyphenation{word list}} 命令来指定使用连字符
的位置,例如 \mintinline{LaTeX}|\hyphenation{FORTRAN Hy-phen-a-tion}|,其中的 \texttt{word list}是不区分大小写的。

\mintinline{LaTeX}|\mbox{text}| 或 \mintinline{LaTeX}|\fbox{text}|:会避免 \texttt{text}被连字符分开。
\mintinline{LaTeX}|\fbox| 比 \mintinline{LaTeX}|\mbox| 多了个可见的框。

\section{预定义好的字符串}
\begin{itemize}
    \item \mintinline{LaTeX}{\today}:\today(打印当天日期)
    \item \mintinline{LaTeX}{\TeX}:\TeX
    \item \mintinline{LaTeX}{\LaTeX}:\LaTeX
    \item \mintinline{LaTeX}{\LaTeXe}:\LaTeXe
\end{itemize}

\section{特殊符号}
\subsection{引号(Quotation Marks)}
\begin{itemize}
    \item 双引号:\mintinline{LaTeX}{``...text...''}
    \item 单引号:\mintinline{LaTeX}{`...text...'}
\end{itemize}

\subsection{短划线(Dashes)和连字符(Hyphens)}
在 \LaTeX 中有下面四种横杠:
\begin{itemize}
    \item \mintinline{LaTeX}{-}:-,连字符(hyphen),用于连接词语
    \item \mintinline{LaTeX}{--}:--,短破折号(en-dash),常用于连接数字表示起止范围
    \item \mintinline{LaTeX}{---}:---,长破折号(em-dash),常用于表示意思的转换
    \item \mintinline{LaTeX}{$-$}:$-$,减号(minus sign)
\end{itemize}

\subsection{波浪线(Tilde)}
\begin{itemize}
    \item \mintinline{LaTeX}{\~{}}:\~{}
    \item \mintinline{LaTeX}{$\sim$}:$\sim$
\end{itemize}

\subsection{斜杠(Slash)}
\begin{itemize}
    \item \mintinline{LaTeX}{read/write}:read/write(不允许用连字符拆分)
    \item \mintinline{LaTeX}{read\slash write}:read\slash write(允许用连字符拆分)
\end{itemize}

\subsection{度(Degree Symbol)}
\begin{itemize}
    \item \mintinline{LaTeX}{$30\,^{\circ}\mathrm{C}$}\footnote{这里的 \mintinline{LaTeX}{\,} 会输出空格}
    :$30\,^{\circ}\mathrm{C}$
    \item \mintinline{LaTeX}{30 \textcelsius{}}:30 \textcelsius{}
    \item \mintinline{LaTeX}{86 \textdegree{}F}:86 \textdegree{}F
\end{itemize}

\subsection{欧元符号}
要想使用欧元符号,需要先在导言区通过 \mintinline{LaTeX}{\usepackage{textcomp}} 命令导入宏包,然后再使用\\
\mintinline{LaTeX}{\texteuro} 命令输出欧元符号。如果所用的字体不包含欧元符号或者想用别的字体的欧元符号,可以通过
\mintinline{LaTeX}{\usepackage[official]{eurosym}} 导入 \texttt{eurosym} 宏包,然后用 \mintinline{LaTeX}{\euro} 输出官方的
欧元符号。用\texttt{gen}来替换\texttt{official}参数可以使用和当前字体匹配的欧元符号。
\begin{itemize}
    \item \mintinline{LaTeX}{\texteuro}:\texteuro
    \item \mintinline{LaTeX}{\euro}:\euro
\end{itemize}

\subsection{省略号(Ellipsis)}
\LaTeX 提供了命令 \mintinline{LaTeX}{\ldots} 来生成省略号,相对于直接输入三个点的方式更为合理。\mintinline{LaTeX}{\ldots} 和 \mintinline{LaTeX}{\dots}
是两个等效的命令。
\begin{itemize}
    \item \mintinline{LaTeX}{Apples, bananas, ...}:Apples, bananas, ...
    \item \mintinline{LaTeX}{Apples, bananas, \ldots}:Apples, bananas, \ldots
    \item \mintinline{LaTeX}{Apples, bananas, \dots}:Apples, bananas, \dots
\end{itemize}

\subsection{连字(Ligatures)}
有些相邻的字母在排版时会连接起来,可以通过 \mintinline{LaTeX}{\mbox{}} 命令避免它们相连。
\begin{itemize}
    \item \mintinline{LaTeX}{ffshfilfluffia}:ffshfilfluffia(相连的情况)
    \item \mintinline{LaTeX}{f\mbox{}fshf\mbox{}ilf\mbox{}luf\mbox{}f\mbox{}ia}:
    f\mbox{}fshf\mbox{}ilf\mbox{}luf\mbox{}f\mbox{}ia(没有相连的情况)
\end{itemize}

\subsection{重音(Accents)符号和特殊符号}
示例代码如下:
\begin{minted}{LaTeX}
    \emph{\=a}  \emph{\'a}  \emph{\v a}  \emph{\` a}

    H\^otel, na\"\i ve, \'el\`eve, \\
    sm\o rrebr\o d, !'Se\~norita!, \\
    Sch\"onbrunner Schlo\ss{}
    Stra\ss e
\end{minted}

示例输出结果如下:
\begin{quote}
    \emph{\=a} \qquad \emph{\'a} \qquad \emph{\v a} \qquad \emph{\` a}

    H\^otel, na\"\i ve, \'el\`eve, \\
    sm\o rrebr\o d, !`Se\~norita!, \\
    Sch\"onbrunner Schlo\ss{}
    Stra\ss e
\end{quote}

重音符号和特殊符号命令列表:
\begin{minted}{text}
    \`o    \'o    \^o    \~o
    \=o    \.o    \"o    \c c
    \u o   \v o   \H o   \c o
    \d o   \b o   \t oo
    \oe    \OE    \ae    \AE
    \aa    \AA
    \o     \O     \l     \L
    \i     \j     !`     ?`
\end{minted}
\newpage
重音符号和特殊符号输出结果列表:
\begin{quote}
    \`o  \qquad  \'o  \qquad  \^o  \qquad  \~o \\
    \=o  \qquad  \.o  \qquad  \"o  \qquad  \c c \\
    \u o \qquad  \v o \qquad  \H o \qquad  \c o \\
    \d o \qquad  \b o \qquad  \t oo \\
    \oe  \qquad  \OE  \qquad  \ae  \qquad  \AE \\
    \aa  \qquad  \AA \\
    \o   \qquad  \O   \qquad  \l   \qquad  \L \\
    \i   \qquad  \j   \qquad  !`   \qquad  ?` \\
\end{quote}

\section{国际语言支持/中文排版支持}
\LaTeX 对其他很多语言提供了支持。\texttt{babel} 宏包可以用于对各种语言进行适配。其他语言暂时也用不到,这里就记录一下如何让 \LaTeX 支持中文。

使用 \LaTeX 排版中文有两种方式,一种是使用 \texttt{xeCJK} 宏包,另一种是使用 \CTeX 宏包和文档类,推荐使用后者。\CTeX 宏包和文档类是对 \texttt{CJK}
和 \texttt{xeCJK} 等宏包的进一步封装。文档类包括 \texttt{ctexart}、\texttt{ctexrep}、\texttt{ctexbook},分别是对 \LaTeX 的三个标准文
档类 \texttt{article}、\texttt{report}、\texttt{book} 的封装,对 \LaTeX 的排版样式做了许多调整,以切合中文排版风格。最新版本的 \CTeX
宏包/文档类甚至支持自动配置字体。

\subsection{\CTeX 的安装}
\CTeX 宏集依赖的宏包和宏集已被最常见的 \TeX 发行版 \TeX Live 和 MiK\TeX 所收录。如果本地安装的 \TeX Live 或 MiK\TeX 不是完全版本,就
需要通过这两个发行版提供的宏包管理器来安装宏包。

\TeX Live 的宏包管理器是 \texttt{tlmgr}。在Linux系统上,一般需要 \texttt{sudo} 权限才能正确地执行 \texttt{tlmgr}的功能。

直接使用 \mintinline{bash}{sudo tlmgr [arg]} 时,可能会提示找不到 \texttt{tlmgr} 或没有这个命令。乍一看,情况比较尴尬:不加
\texttt{sudo} 没有权限,加了 \texttt{sudo} 反而找不到命令了。经过上网搜索,在一个帖子\footnote{\emph{sudo does not find tlmgr},
https://tex.stackexchange.com/questions/203874/sudo-does-not-find-tlmgr} 里找到了解决方法。原来,\texttt{sudo} 有一种内置的
保护机制,只会使用安全的环境变量 \texttt{PATH}。如果 \TeX Live 的路径不在 \texttt{sudo} 的安全环境变量内,它就找不到相关的命令。可以
在终端执行 \mintinline{bash}{sudo gedit /etc/sudoers},然后将\TeX Live的路径添加到 \texttt{sudo} 的 \texttt{secure\_path}中。
在我的 Ubuntu 16.04 上,添加后结果如下(后面的原有路径省略,不同路径用 \texttt{:} 隔开):
\begin{minted}{text}
Defaults secure_path="/usr/local/texlive/2016/bin/x86_64-linux:/usr/local/sbin:..."
\end{minted}

不能用 \texttt{sudo} 执行 \texttt{tlmgr} 的问题解决后,在终端中依次执行以下命令,以更新 \texttt{tlmgr} 宏包管理器、已安装的所有宏包、安装 \CTeX
宏集。
\begin{minted}{bash}
    sudo tlmgr update --self
    sudo tlmgr update --all
    sudo tlmgr install ctex
\end{minted}

\subsection{使用 \CTeX 文档类}
\CTeX 宏集提供了四个中文文档类:\texttt{ctexart}、\texttt{ctexrep}、\texttt{ctexbook} 和 \texttt{ctexbeamer},分别对应 \LaTeX
的标准文档类 \texttt{article}、\texttt{report}、\texttt{book} 和 \texttt{beamer}。使用它们的时候,需要将涉及到的所有源文件使用
UTF-8 编码保存。

下面是使用 \texttt{ctexart} 文档类编写的一个例子:
\begin{minted}{LaTeX}
    \documentclass[UTF8]{ctexart}
    \begin{document}
    中文文档类测试。你需要将所有源文件保存为 UTF-8 编码。
    你可以使用 XeLaTeX、LuaLaTeX 或 upLaTeX 编译,也可以使用 (pdf)LaTeX 编译。
    推荐使用 XeLaTeX 或 LuaLaTeX 编译。
    \end{document}
\end{minted}

\CTeX 预定义的字库中的中文字体已经基本够用,包括{\songti 宋体}(\mintinline{LaTeX}{\songti})、{\heiti 黑体}
(\mintinline{LaTeX}{\heiti})、{\kaishu 楷书}(\mintinline{LaTeX}{\kaishu})、{\fangsong 仿宋}
(\mintinline{LaTeX}{\fangsong})等。更多 \CTeX 的使用参考《\CTeX 宏集手册》
\footnote{《\CTeX 宏集手册》,http://mirror.unl.edu/ctan/language/chinese/ctex/ctex.pdf}。

\section{单词之间的空格}
为了使输出更美观、更具可读性,\LaTeX 可能会在不同单词之间或句子末尾插入更多空格。\LaTeX 默认句子以句点(periods)、问号(question marks)
或者感叹号(exclamation marks)结尾。但是如果句点跟在一个大写字母后面,它不会认为这是句子结尾,因为大写字母后面跟句点往往是缩略词。

用户可以通过具体的命令来改变上面的默认设定。一个斜杠跟一个空格会产生一个不会被扩大的空格;一个波浪线(\texttt{$\sim$})会产生一个既不能被扩
大、也不能从这里断行的空格;在句点前使用 \mintinline{LaTeX}{\@} 命令,不管这个句点是不是跟在大写字母后面,都会指定这个句子到句点就结束。
使用 \mintinline{LaTeX}{\frenchspacing} 命令可以强制不在一个句子后面插入多余的空格。如果使用 \mintinline{LaTeX}{\frenchspacing}
命令就没必要再用 \mintinline{LaTeX}{\@} 了。


代码示例:
\begin{minted}{LaTeX}
    Mr.~Smith was happy to see her\\
    cf.~Fig.~5\\
    I like BASIC\@. What about you?
\end{minted}

示例输出:
\begin{quote}
    Mr.~Smith was happy to see her\\
    cf.~Fig.~5\\
    I like BASIC\@. What about you?
\end{quote}
\newpage

\section{标题、章、节}
文档类 \texttt{article} 中有以下几种分层次结构的命令:
\begin{minted}{LaTeX}
    \section{...}
    \subsection{...}
    \subsubsection{...}
    \paragraph{...}
    \subparagraph{...}
\end{minted}

\mintinline{LaTeX}{\part{...}} 命令也可以把文档分为多个部分,但它不会影响 \texttt{section} 和 \texttt{chapter} 的编号。

和 \texttt{article} 文档类相比,在 \texttt{report} 和 \texttt{book} 中, 可以使用 \mintinline{LaTeX}{\chapter{...}}。

由于 \texttt{article} 文档类中不包含 \texttt{chapter},所以可以很方便地把 \texttt{article} 作为 \texttt{chapter} 插入
\texttt{book} 文档类。章节空隙、编号等由 \LaTeX 自动完成。

下面是两个比较特殊的情况:
\begin{itemize}
    \item \mintinline{LaTeX}{\part} 命令不会影响 \texttt{chapter} 或 \texttt{section} 的编号
    \item \mintinline{LaTeX}{\appendix} 命令没有任何参数,会把 \texttt{chapter} (对于 \texttt{report}、\texttt{book})
    或 \texttt{section} (对于 \texttt{article})的数字编号转换成字母编号。
\end{itemize}


\end{document}
