%-*- coding: UTF-8 -*-
% MathGraphs.tex
%
\documentclass[UTF8]{ctexart}
\usepackage{geometry}
\geometry{a4paper, centering, scale=0.8}
\usepackage{minted}
\usepackage{float}
\usepackage{amsmath}
\usepackage{hyperref}

\title{\heiti 第5章 \quad 数学插图}
\author{\kaishu Du Ang \\ \texttt{du2ang233@gmail.com} }
\date{\today}


\begin{document}
\maketitle

\tableofcontents

\newpage

\section{概述}
除了排版文字,\LaTeX 也支持用代码表示图形。\LaTeX 提供了原始的 \texttt{picture} 环境,能够绘制一些基本的图形如
点、线、矩形、圆等等。不过受制于 \LaTeX 本身,它的绘图功能极为有限,效果也不够美观。

不同的扩展极大地丰富了 \LaTeX 的图形功能,TikZ 就是其中之一。一些特殊的绘图,如交换图、树状图甚至分子式和电路图也能
够通过代码绘制,不过过于复杂。

现在流行的绘图代码有以下几种:
\begin{itemize}
    \item PSTricks \\ 以 PostScript 语言的功能为基础的绘图宏包,具有优秀的绘图能力。它对老式的
    \texttt{latex + dvips} 编译命令支持最好,而现在的几种编译命令下使用起来都不够方便。
    \item TikZ \& pgf \\ 德国的 Till Tantau 在开发著名的 \LaTeX 幻灯片文档类 \texttt{beamer} 时一并开发了绘
    图宏包 \texttt{pgf},目的是令其能够在 \texttt{pdflatex} 或 \texttt{xelatex} 等不同的编译命令下都能使用。
    \texttt{TikZ} 是在 \texttt{pgf} 基础上封装的一个宏包,采用了类似 METAPOST 的语法,提供了方便的绘图命令。
    \item METAPOST \& Asymptote \\ METAPOST 脱胎于高德纳为 \TeX{} 配套开发的字体生成程序 METAFONT,具有优秀
    的绘图能力,并能够调用 \TeX{} 引擎向图片中插入文字和公式。Asymptote 在 METAPOST 的基础上更进一步,具有一定的
    类似 C 语言的编程能力,支持三维图形的绘制。
\end{itemize}

它们往往需要把代码写在单独的文件里,用特定的工具去编译,也可以借助特殊的宏包在 \LaTeX 代码里直接使用。

\section{\texttt{picture} 环境}
\subsection{基本命令}
\texttt{picture} 环境可以通过以下命令创建:
\begin{minted}{LaTeX}
    \begin{picture}(x,y)...\end{picture}
    或
    \begin{picture}(x,y)(x0,y0)...\end{picture}
\end{minted}

数字 \emph{x}、\emph{y}、\emph{x0}、\emph{y0} 和 \mintinline{LaTeX}{\unitlength} 有关。
\mintinline{LaTeX}{\unitlength} 默认值是 \texttt{1pt},可以通过
\mintinline{LaTeX}{\setlength{\unitlength}{1.2cm}} 命令设置。

大多数的绘图命令有两种形式:
\begin{minted}{LaTeX}
    \put{x,y}{object}
    或
    \multiput(x,y)(x_delta, y_delta){n}{object}
\end{minted}

但是贝塞尔曲线(B\'ezier curves)是个例外,通过
\mintinline{LaTeX}{\qbezier(x_1, y_1)(x_2, y_2)(x_3, y_3)} 命令绘制。

\subsection{线段(Line Segments)}
线段命令:\mintinline{LaTeX}{\put(x,y){\line(x1,y1){length}}}

\mintinline{LaTeX}{\line} 命令有两个参数:1. 方向向量,2. 长度

方向向量的取值只限于 $\{-6, -5, ..., 5, 6\}$ 这些整数,并且需要互素。

下面的示例展示了第一象限中所有 25 种可能的斜线段。线段的长度跟单位长度 \mintinline{LaTeX}{\unitlength} 有关。

示例代码:
\begin{minted}{LaTeX}
    \setlength{\unitlength}{5cm}
    \begin{picture}(1, 1)
        \put(0, 0){\line(0, 1){1}}
        \put(0, 0){\line(1, 0){1}}
        \put(0, 0){\line(1, 1){1}}
        \put(0, 0){\line(1, 2){.5}}
        \put(0, 0){\line(1, 3){.3333}}
        \put(0, 0){\line(1, 4){.25}}
        \put(0, 0){\line(1, 5){.2}}
        \put(0, 0){\line(1, 6){.166667}}
        \put(0, 0){\line(2, 1){1}}
        \put(0, 0){\line(2, 3){.66667}}
        \put(0, 0){\line(2, 5){.4}}
        \put(0, 0){\line(3, 1){1}}
        \put(0, 0){\line(3, 2){1}}
        \put(0, 0){\line(3, 4){.75}}
        \put(0, 0){\line(3, 5){.6}}
        \put(0, 0){\line(4, 1){1}}
        \put(0, 0){\line(4, 3){1}}
        \put(0, 0){\line(4, 5){.8}}
        \put(0, 0){\line(5, 1){1}}
        \put(0, 0){\line(5, 2){1}}
        \put(0, 0){\line(5, 3){1}}
        \put(0, 0){\line(5, 4){1}}
        \put(0, 0){\line(5, 6){.833333}}
        \put(0, 0){\line(6, 1){1}}
        \put(0, 0){\line(6, 1){1}}
    \end{picture}
\end{minted}

示例输出:

\setlength{\unitlength}{5cm}
\begin{picture}(1, 1)
    \put(0, 0){\line(0, 1){1}}
    \put(0, 0){\line(1, 0){1}}
    \put(0, 0){\line(1, 1){1}}
    \put(0, 0){\line(1, 2){.5}}
    \put(0, 0){\line(1, 3){.3333}}
    \put(0, 0){\line(1, 4){.25}}
    \put(0, 0){\line(1, 5){.2}}
    \put(0, 0){\line(1, 6){.166667}}
    \put(0, 0){\line(2, 1){1}}
    \put(0, 0){\line(2, 3){.66667}}
    \put(0, 0){\line(2, 5){.4}}
    \put(0, 0){\line(3, 1){1}}
    \put(0, 0){\line(3, 2){1}}
    \put(0, 0){\line(3, 4){.75}}
    \put(0, 0){\line(3, 5){.6}}
    \put(0, 0){\line(4, 1){1}}
    \put(0, 0){\line(4, 3){1}}
    \put(0, 0){\line(4, 5){.8}}
    \put(0, 0){\line(5, 1){1}}
    \put(0, 0){\line(5, 2){1}}
    \put(0, 0){\line(5, 3){1}}
    \put(0, 0){\line(5, 4){1}}
    \put(0, 0){\line(5, 6){.833333}}
    \put(0, 0){\line(6, 1){1}}
    \put(0, 0){\line(6, 1){1}}
\end{picture}

\subsection{箭头(Arrows)}
箭头命令:\mintinline{LaTeX}{\put(x,y){\verctor(x1,y1){length}}}

对于箭头,方向向量从 $\{-4, -3, ..., 3, 4\}$ 中取值,并且也需要互素。\mintinline{LaTeX}{\thicklines} 命令
和 \mintinline{LaTeX}{\thinlines} 命令可以控制箭头的粗细。

示例代码:
\begin{minted}{LaTeX}
    \setlength{\unitlength}{0.75mm}
    \begin{picture}(60, 40)
        \put(30, 20){\vector(1, 0){30}}
        \put(30, 20){\vector(4, 1){20}}
        \put(30, 20){\vector(3, 1){25}}
        \put(30, 20){\vector(2, 1){30}}
        \put(30, 20){\vector(1, 2){10}}
        \thicklines
        \put(30, 20){\vector(-4, 1){30}}
        \put(30, 20){\vector(-1, 4){5}}
        \thinlines
        \put(30, 20){\vector(-1, -1){5}}
        \put(30, 20){\vector(-1, -4){5}}
    \end{picture}
\end{minted}

示例输出:

\setlength{\unitlength}{0.75mm}
\begin{picture}(60, 40)
    \put(30, 20){\vector(1, 0){30}}
    \put(30, 20){\vector(4, 1){20}}
    \put(30, 20){\vector(3, 1){25}}
    \put(30, 20){\vector(2, 1){30}}
    \put(30, 20){\vector(1, 2){10}}
    \thicklines
    \put(30, 20){\vector(-4, 1){30}}
    \put(30, 20){\vector(-1, 4){5}}
    \thinlines
    \put(30, 20){\vector(-1, -1){5}}
    \put(30, 20){\vector(-1, -4){5}}
\end{picture}

\subsection{圆(Circles)}
\mintinline{LaTeX}{\put(x,y){\circle{diameter}}} 命令会以 $(x,y)$ 为圆心、以 \emph{diameter} 为半径的圆。
\texttt{picture} 环境允许最大的直径大约为 14mm。\mintinline{LaTeX}{\circle*} 命令可以画实心的圆盘。

示例代码:
\begin{minted}{LaTeX}
    \setlength{\unitlength}{1mm}
    \begin{picture}(60, 40)
        \put(20, 30){\circle{1}}
        \put(20, 30){\circle{2}}
        \put(20, 30){\circle{3}}
        \put(20, 30){\circle{4}}
        \put(20, 30){\circle{8}}
        \put(20, 30){\circle{16}}
        \put(20, 30){\circle{32}}

        \put(40, 30){\circle{1}}
        \put(40, 30){\circle{2}}
        \put(40, 30){\circle{3}}
        \put(40, 30){\circle{4}}
        \put(40, 30){\circle{5}}
        \put(40, 30){\circle{6}}
        \put(40, 30){\circle{7}}
        \put(40, 30){\circle{8}}
        \put(40, 30){\circle{9}}
        \put(40, 30){\circle{10}}
        \put(40, 30){\circle{11}}
        \put(40, 30){\circle{12}}
        \put(40, 30){\circle{13}}
        \put(40, 30){\circle{14}}

        \put(15, 10){\circle*{1}}
        \put(20, 10){\circle*{2}}
        \put(25, 10){\circle*{3}}
        \put(30, 10){\circle*{4}}
        \put(35, 10){\circle*{5}}
    \end{picture}
\end{minted}

示例输出:

\setlength{\unitlength}{1mm}
\begin{picture}(60, 40)
    \put(20, 30){\circle{1}}
    \put(20, 30){\circle{2}}
    \put(20, 30){\circle{3}}
    \put(20, 30){\circle{4}}
    \put(20, 30){\circle{8}}
    \put(20, 30){\circle{16}}
    \put(20, 30){\circle{32}}

    \put(40, 30){\circle{1}}
    \put(40, 30){\circle{2}}
    \put(40, 30){\circle{3}}
    \put(40, 30){\circle{4}}
    \put(40, 30){\circle{5}}
    \put(40, 30){\circle{6}}
    \put(40, 30){\circle{7}}
    \put(40, 30){\circle{8}}
    \put(40, 30){\circle{9}}
    \put(40, 30){\circle{10}}
    \put(40, 30){\circle{11}}
    \put(40, 30){\circle{12}}
    \put(40, 30){\circle{13}}
    \put(40, 30){\circle{14}}

    \put(15, 10){\circle*{1}}
    \put(20, 10){\circle*{2}}
    \put(25, 10){\circle*{3}}
    \put(30, 10){\circle*{4}}
    \put(35, 10){\circle*{5}}
\end{picture}

\subsection{文字和公式}
示例代码:
\begin{minted}{LaTeX}
    \setlength{\unitlength}{0.8cm}
    \begin{picture}(6, 5)
        \thicklines
        \put(1, 0.5){\line(2, 1){3}}
        \put(4, 2){\line(-2, 1){2}}
        \put(2, 3){\line(-2, -5){1}}
        \put(0.7, 0.3){$A$}
        \put(4.05, 1.9){$B$}
        \put(1.7, 2.95){$C$}
        \put(3.1, 2.5){$a$}
        \put(1.3, 1.7){$b$}
        \put(2.5, 1.05){$c$}
        \put(0.3, 4){$F=\sqrt{s(s-a)(s-b)(s-c)}$}
        \put(3.5, 0.4){$\displaystyle s:=\frac{a+b+c}{2}$}
    \end{picture}
\end{minted}

示例输出:

\setlength{\unitlength}{0.8cm}
\begin{picture}(6, 5)
    \thicklines
    \put(1, 0.5){\line(2, 1){3}}
    \put(4, 2){\line(-2, 1){2}}
    \put(2, 3){\line(-2, -5){1}}
    \put(0.7, 0.3){$A$}
    \put(4.05, 1.9){$B$}
    \put(1.7, 2.95){$C$}
    \put(3.1, 2.5){$a$}
    \put(1.3, 1.7){$b$}
    \put(2.5, 1.05){$c$}
    \put(0.3, 4){$F=\sqrt{s(s-a)(s-b)(s-c)}$}
    \put(3.5, 0.4){$\displaystyle s:=\frac{a+b+c}{2}$}
\end{picture}

\subsection{\texttt{\textbackslash multiput} 和 \texttt{\textbackslash linethickness}}
\mintinline{LaTeX}{\multiput(x, y)(delta_x, delta_y){n}{object}} 命令有 4 个参数:起点,从一个物体平移到
另一个的平移向量,物体的数量,待画物体。\mintinline{LaTeX}{\linethickness} 命令只适用于横线和竖线,不能用于斜线
和圆,但是能用于二次贝塞尔曲线。

示例代码:
\begin{minted}{LaTeX}
    \setlength{\unitlength}{2mm}
    \begin{picture}(30, 20)
    \linethickness{0.075mm}
    \multiput(0, 0)(1, 0){26}{\line(0, 1){20}}
    \multiput(0, 0)(0, 1){21}{\line(1, 0){25}}
    \linethickness{0.15mm}
    \multiput(0, 0)(5, 0){6}{\line(0, 1){20}}
    \multiput(0, 0)(0, 5){5}{\line(1, 0){25}}
    \linethickness{0.3mm}
    \multiput(5, 0)(10, 0){2}{\line(0, 1){20}}
    \multiput(0, 5)(0, 10){2}{\line(1, 0){25}}
    \end{picture}
\end{minted}

示例输出:

\setlength{\unitlength}{2mm}
\begin{picture}(30, 20)
    \linethickness{0.075mm}
    \multiput(0, 0)(1, 0){26}{\line(0, 1){20}}
    \multiput(0, 0)(0, 1){21}{\line(1, 0){25}}
    \linethickness{0.15mm}
    \multiput(0, 0)(5, 0){6}{\line(0, 1){20}}
    \multiput(0, 0)(0, 5){5}{\line(1, 0){25}}
    \linethickness{0.3mm}
    \multiput(5, 0)(10, 0){2}{\line(0, 1){20}}
    \multiput(0, 5)(0, 10){2}{\line(1, 0){25}}
\end{picture}

\subsection{椭圆(Ovals)}
\mintinline{LaTeX}{\put(x, y){\oval(w, h)}} 或 \mintinline{LaTeX}{\put(x, y){\oval(w, h)[position]}}
命令可以产生一个以 $(x, y)$ 为中心、宽 $w$、高 $h$ 的椭圆。可选参数 \emph{position} 可以为 \texttt{b}、
\texttt{t}、\texttt{l}、\texttt{r} 或其组合,分别代表上(“top”)、下(“bottom”)、左(“left”)、
右(“right”)。

一方面,线宽可以通过 \mintinline{LaTeX}{\linethickness} 命令指定;另一方面,可以通过
\mintinline{LaTeX}{\thinlines} 和 \mintinline{LaTeX}{\thicklines} 命令控制。
\mintinline{LaTeX}{\linethickness} 仅可用于横线、竖线和贝塞尔曲线,\mintinline{LaTeX}{\thinlines} 和
\mintinline{LaTeX}{\thicklines} 可以用于斜线段、圆和椭圆。

示例代码:
\begin{minted}{LaTeX}
    \setlength{\unitlength}{0.75cm}
    \begin{picture}(6, 4)
        \linethickness{0.075mm}
        \multiput(0, 0)(1, 0){7}{\line(0, 1){4}}
        \multiput(0, 0)(0, 1){5}{\line(1, 0){6}}
        \thicklines
        \put(2, 3){\oval(3, 1.8)}
        \thinlines
        \put(3, 2){\oval(3, 1.8)}
        \thicklines
        \put(2, 1){\oval(3, 1.8)[tl]}
        \put(4, 1){\oval(3, 1.8)[b]}
        \put(4, 3){\oval(3, 1.8)[r]}
        \put(3, 1.5){\oval(1.8, 0.4)}
    \end{picture}
\end{minted}

示例输出:

\setlength{\unitlength}{0.75cm}
\begin{picture}(6, 4)
    \linethickness{0.075mm}
    \multiput(0, 0)(1, 0){7}{\line(0, 1){4}}
    \multiput(0, 0)(0, 1){5}{\line(1, 0){6}}
    \thicklines
    \put(2, 3){\oval(3, 1.8)}
    \thinlines
    \put(3, 2){\oval(3, 1.8)}
    \thicklines
    \put(2, 1){\oval(3, 1.8)[tl]}
    \put(4, 1){\oval(3, 1.8)[b]}
    \put(4, 3){\oval(3, 1.8)[r]}
    \put(3, 1.5){\oval(1.8, 0.4)}
\end{picture}

\subsection{图片框(Picture Boxes)}
\begin{enumerate}
    \item 通过 \mintinline{LaTeX}{\newsavebox{name}} 来声明一个图片框;
    \item 然后用 \mintinline{LaTeX}{\savebox{name}(width, height)[position]{content}} 命令进行定义;
    \item 最后用 \mintinline{LaTeX}{\put(x, y){\usebox{name}}} 命令使用定义好的图片框。
\end{enumerate}

上面命令中的可选参数 \emph{position} 用于定义 savebox 中的定位点(anchor point)。

示例代码:
\begin{minted}{LaTeX}
    \setlength{\unitlength}{0.5mm}
    \begin{picture}(120, 168)
        \newsavebox{\foldera}
        \savebox{\foldera}(40, 32)[bl]{
            % definition of foldera
            \multiput(0, 0)(0, 28){2}{\line(1, 0){40}}
            \multiput(0, 0)(40, 0){2}{\line(0, 1){28}}
            \put(1, 28){\oval(2, 2)[tl]}
            \put(1, 29){\line(1, 0){5}}
            \put(9, 29){\oval(6, 6)[tl]}
            \put(9, 32){\line(1, 0){8}}
            \put(17, 29){\oval(6, 6)[tr]}
            \put(20, 29){\line(1, 0){19}}
            \put(39, 28){\oval(2, 2)[tr]}
        }
        \newsavebox{\folderb}
        \savebox{\folderb}(40, 32)[l]{
            % definition of folderb
            \put(0, 14){\line(1, 0){8}}
            \put(8, 0){\usebox{\foldera}}
        }
        \put(34, 26){\line(0, 1){102}}
        \put(14, 128){\usebox{\foldera}}
        \multiput(34, 86)(0, -37){3}{\usebox{\folderb}}
    \end{picture}
\end{minted}

示例输出:

\setlength{\unitlength}{0.5mm}
\begin{picture}(120, 168)
    \newsavebox{\foldera}
    \savebox{\foldera}(40, 32)[bl]{
        % definition of foldera
        \multiput(0, 0)(0, 28){2}{\line(1, 0){40}}
        \multiput(0, 0)(40, 0){2}{\line(0, 1){28}}
        \put(1, 28){\oval(2, 2)[tl]}
        \put(1, 29){\line(1, 0){5}}
        \put(9, 29){\oval(6, 6)[tl]}
        \put(9, 32){\line(1, 0){8}}
        \put(17, 29){\oval(6, 6)[tr]}
        \put(20, 29){\line(1, 0){19}}
        \put(39, 28){\oval(2, 2)[tr]}
    }
    \newsavebox{\folderb}
    \savebox{\folderb}(40, 32)[l]{
        % definition of folderb
        \put(0, 14){\line(1, 0){8}}
        \put(8, 0){\usebox{\foldera}}
    }
    \put(34, 26){\line(0, 1){102}}
    \put(14, 128){\usebox{\foldera}}
    \multiput(34, 86)(0, -37){3}{\usebox{\folderb}}
\end{picture}

\subsection{二次贝塞尔曲线(Quadratic B\'ezier Curves)}
贝塞尔曲线(B\'ezier curves)可通过 \mintinline{LaTeX}{\qbezier(x_1, y_1)(x_2, y_2)(x_3, y_3)} 命令绘制。

用 $P_1 = (x_1, y_1)$, $P_2 = (x_2, y_2)$ 表示两个端点,用 $m_1$、$m_2$ 分别表示二次贝塞尔曲线的两个斜率。中
间控制点 $S = (x, y)$ 由下面的方程得到:
\begin{equation}
    \label{eq:bezier}
    \left\{
    \begin{array}{rcl}
        rclx & = & \frac{\displaystyle m_2 x_2 - m_1 x_1 - (y_2 - y_1)}{\displaystyle m_2 - m_1}, \\
        y & = & y_i + m_i (x - x_i) \qquad (i = 1, 2).
    \end{array} \right.
\end{equation}

示例代码:
\begin{minted}{LaTeX}
    \setlength{\unitlength}{0.8cm}
    \begin{picture}(6, 4)
        \linethickness{0.075mm}
        \multiput(0, 0)(1, 0){7}{\line(0, 1){4}}
        \multiput(0, 0)(0, 1){5}{\line(1, 0){6}}
        \thicklines
        \put(0.5, 0.5){\line(1, 5){0.5}}
        \put(1, 3){\line(4, 1){2}}
        \qbezier(0.5, 0.5)(1, 3)(3, 3.5)
        \thinlines
        \put(2.5, 2){\line(2, -1){3}}
        \put(5.5, 0.5){\line(-1, 5){0.5}}
        \linethickness{1mm}
        \qbezier(2.5, 2)(5.5, 0.5)(5, 3)
        \thinlines
        \qbezier(4, 2)(4, 3)(3, 3)
        \qbezier(3, 3)(2, 3)(2, 2)
        \qbezier(2, 2)(2, 1)(3, 1)
        \qbezier(3, 1)(4, 1)(4, 2)
    \end{picture}
\end{minted}

示例输出:

\setlength{\unitlength}{0.8cm}
\begin{picture}(6, 4)
    \linethickness{0.075mm}
    \multiput(0, 0)(1, 0){7}{\line(0, 1){4}}
    \multiput(0, 0)(0, 1){5}{\line(1, 0){6}}
    \thicklines
    \put(0.5, 0.5){\line(1, 5){0.5}}
    \put(1, 3){\line(4, 1){2}}
    \qbezier(0.5, 0.5)(1, 3)(3, 3.5)
    \thinlines
    \put(2.5, 2){\line(2, -1){3}}
    \put(5.5, 0.5){\line(-1, 5){0.5}}
    \linethickness{1mm}
    \qbezier(2.5, 2)(5.5, 0.5)(5, 3)
    \thinlines
    \qbezier(4, 2)(4, 3)(3, 3)
    \qbezier(3, 3)(2, 3)(2, 2)
    \qbezier(2, 2)(2, 1)(3, 1)
    \qbezier(3, 1)(4, 1)(4, 2)
\end{picture}

\subsection{垂曲线(Catenary)}

示例代码:
\begin{minted}{LaTeX}
    \setlength{\unitlength}{1cm}
    \begin{picture}(4.3, 3.6)(-2.5, -0.25)
        \put(-2,  0){\vector(1,  0){4.4}}
        \put(2.45, -.05){$x$}
        \put(0, 0){\vector(0, 1){3.2}}
        \put(0, 3.35){\makebox(0, 0){$y$}}
        \qbezier(0.0, 0.0)(1.2384, 0.0)(2.0, 2.7622)
        \qbezier(0.0, 0.0)(-1.2384, 0.0)(-2.0, 2.7622)
        \linethickness{.075mm}
        \multiput(-2, 0)(1, 0){5}{\line(0, 1){3}}
        \multiput(-2, 0)(0, 1){4}{\line(1, 0){4}}
        \linethickness{.2mm}
        \put( .3, .12763){\line(1, 0){.4}}
        \put(.5, -.07237){\line(0, 1){.4}}
        \put(-.7, .12763){\line(1, 0){.4}}
        \put(-.5, -.07237){\line(0, 1){.4}}
        \put(.8, .54308){\line(1, 0){.4}}
        \put(1, .34308){\line(0, 1){.4}}
        \put(-1.2, .54308){\line(1, 0){.4}}
        \put(-1, .34308){\line(0, 1){.4}}
        \put(1.3, 1.35241){\line(1, 0){.4}}
        \put(1.5, 1.15241){\line(0, 1){.4}}
        \put(-1.7, 1.35241){\line(1, 0){.4}}
        \put(-1.5, 1.15241){\line(0, 1){.4}}
        \put(-2.5, -0.25){\circle*{0.2}}
    \end{picture}
\end{minted}

示例输出:

\setlength{\unitlength}{1cm}
\begin{picture}(4.3, 3.6)(-2.5, -0.25)
    \put(-2,  0){\vector(1,  0){4.4}}
    \put(2.45, -.05){$x$}
    \put(0, 0){\vector(0, 1){3.2}}
    \put(0, 3.35){\makebox(0, 0){$y$}}
    \qbezier(0.0, 0.0)(1.2384, 0.0)(2.0, 2.7622)
    \qbezier(0.0, 0.0)(-1.2384, 0.0)(-2.0, 2.7622)
    \linethickness{.075mm}
    \multiput(-2, 0)(1, 0){5}{\line(0, 1){3}}
    \multiput(-2, 0)(0, 1){4}{\line(1, 0){4}}
    \linethickness{.2mm}
    \put( .3, .12763){\line(1, 0){.4}}
    \put(.5, -.07237){\line(0, 1){.4}}
    \put(-.7, .12763){\line(1, 0){.4}}
    \put(-.5, -.07237){\line(0, 1){.4}}
    \put(.8, .54308){\line(1, 0){.4}}
    \put(1, .34308){\line(0, 1){.4}}
    \put(-1.2, .54308){\line(1, 0){.4}}
    \put(-1, .34308){\line(0, 1){.4}}
    \put(1.3, 1.35241){\line(1, 0){.4}}
    \put(1.5, 1.15241){\line(0, 1){.4}}
    \put(-1.7, 1.35241){\line(1, 0){.4}}
    \put(-1.5, 1.15241){\line(0, 1){.4}}
    \put(-2.5, -0.25){\circle*{0.2}}
\end{picture}

在上面的示例中,每一半垂曲线 $y = \cosh x - 1$ 都是由二次贝塞尔曲线近似的。右半边曲线的端点是 $(2, 2.7622)$,该
点的斜率为 $m = 3.6269$。再次利用公式~\eqref{eq:bezier},可以计算得到中间控制点,分别是 $(1.2384, 0)$ 和
$(-1.2384, 0)$。图中十字标出的是真实垂曲线上的点,和近似的点误差很小,少于百分之一。

上面的示例也展示如何使用 \mintinline{LaTeX}{\begin{picture}} 命令的可选参数:
\begin{minted}{LaTeX}
    \begin{picture}(4.3, 3.6)(-2.5, -0.25)
\end{minted}

\subsection{狭义相对论速度(Rapidity in the Special Theory of Relativity)}
示例代码:
\begin{minted}{LaTeX}
    \setlength{\unitlength}{0.8cm}
    \begin{picture}(6, 4)(-3, -2)
        \put(-2.5, 0){\vector(1, 0){5}}
        \put(2.7, -0.1){$\chi$}
        \put(0, -1.5){\vector(0, 1){3}}
        \multiput(-2.5, -1)(0.4, 0){13}{\line(1, 0){0.2}}
        \put(0.2, 1.4){$\beta = v / c = \tanh \chi$}
        \qbezier(0, 0)(0.8853, 0.8853)(2, 0.9640)
        \qbezier(0, 0)(-0.8853, -0.8853)(-2, -0.9640)
        \put(-3, -2){\circle*{0.2}}
    \end{picture}
\end{minted}

示例输出:

\setlength{\unitlength}{0.8cm}
\begin{picture}(6, 4)(-3, -2)
    \put(-2.5, 0){\vector(1, 0){5}}
    \put(2.7, -0.1){$\chi$}
    \put(0, -1.5){\vector(0, 1){3}}
    \multiput(-2.5, -1)(0.4, 0){13}{\line(1, 0){0.2}}
    \put(0.2, 1.4){$\beta = v / c = \tanh \chi$}
    \qbezier(0, 0)(0.8853, 0.8853)(2, 0.9640)
    \qbezier(0, 0)(-0.8853, -0.8853)(-2, -0.9640)
    \put(-3, -2){\circle*{0.2}}
\end{picture}

两条贝塞尔曲线的控制点是由公式~\eqref{eq:bezier} 计算得到的。正半支曲线由 $P_1 = (0, 0)$,$m_1 = 1$,
$P_2 = (2, \tanh2)$,$m_2 = 1 / \cosh ^2 2$ 决定。



\end{document}
