%-*- coding: UTF-8 -*-
% MathGraphs.tex
%
\documentclass[UTF8]{ctexart}
\usepackage{geometry}
\geometry{a4paper, centering, scale=0.8}
\usepackage{minted}
\usepackage{float}
\usepackage{hyperref}

\title{\heiti 第5章 \quad 数学插图}
\author{\kaishu Du Ang \\ \texttt{du2ang233@gmail.com} }
\date{\today}


\begin{document}
\maketitle

\tableofcontents

\newpage

\section{概述}
除了排版文字,\LaTeX 也支持用代码表示图形。不同的扩展已经极大地丰富了 \LaTeX 的图形功能,TikZ 就是其中之一。一些特
殊的绘图,如交换图、树状图甚至分子式和电路图也能够通过代码绘制,不过过于复杂,这里只介绍 TikZ。

\LaTeX 提供了原始的 \texttt{picture} 环境,能够绘制一些基本的图形如点、线、矩形、圆等等。不过受制于 \LaTeX 本身,
它的绘图功能极为有限,效果也不够美观。

现在流行的绘图代码有以下几种:
\begin{itemize}
    \item PSTricks \\ 以 PostScript 语言的功能为基础的绘图宏包,具有优秀的绘图能力。它对老式的
    \texttt{latex + dvips} 编译命令支持最好,而现在的几种编译命令下使用起来都不够方便。
    \item TikZ \& pgf \\ 德国的 Till Tantau 在开发著名的 \LaTeX 幻灯片文档类 \texttt{beamer} 时一并开发了绘
    图宏包 \texttt{pgf},目的是令其能够在 \texttt{pdflatex} 或 \texttt{xelatex} 等不同的编译命令下都能使用。
    \texttt{TikZ} 是在 \texttt{pgf} 基础上封装的一个宏包,采用了类似 METAPOST 的语法,提供了方便的绘图命令。
    \item METAPOST \& Asymptote \\ METAPOST 脱胎于高德纳为 \TeX{} 配套开发的字体生成程序 METAFONT,具有优秀
    的绘图能力,并能够调用 \TeX{} 引擎向图片中插入文字和公式。Asymptote 在 METAPOST 的基础上更进一步,具有一定的
    类似 C 语言的编程能力,支持三维图形的绘制。
\end{itemize}

它们往往需要把代码写在单独的文件里,用特定的工具去编译,也可以借助特殊的宏包在 \LaTeX 代码里直接使用。

\section{\texttt{picture} 环境}
\subsection{基本命令}
\texttt{picture} 环境可以通过以下命令创建:
\begin{minted}{LaTeX}
    \begin{picture}(x,y)...\end{picture}
    或
    \begin{picture}(x,y)(x0,y0)...\end{picture}
\end{minted}

数字 \emph{x}、\emph{y}、\emph{x0}、\emph{y0} 和 \mintinline{LaTeX}{\unitlength} 有关。
\mintinline{LaTeX}{\unitlength} 默认值是 \texttt{1pt},可以通过
\mintinline{LaTeX}{\setlength{\unitlength}{1.2cm}} 命令设置。

大多数的绘图命令有两种形式:
\begin{minted}{LaTeX}
    \put{x,y}{object}
    或
    \multiput(x,y)(x_delta, y_delta){n}{object}
\end{minted}

但是贝塞尔曲线(B\'ezier curves)是个例外,通过
\mintinline{LaTeX}{\qbezier(x_1, y_1)(x_2, y_2)(x_3, y_3)} 命令绘制。






\end{document}
