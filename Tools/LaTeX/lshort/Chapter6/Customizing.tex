%-*- coding: UTF-8 -*-
% Customizing.tex
%
\documentclass[UTF8]{ctexart}
\usepackage{geometry}
\geometry{a4paper, centering, scale=0.8}
\usepackage{minted}
\usepackage{hyperref}

\title{\heiti 第6章 \quad \LaTeX 的私人订制}
\author{\kaishu Du Ang \\ \texttt{du2ang233@gmail.com} }
\date{\today}


\begin{document}
\maketitle

\tableofcontents

\newpage

之前学到的一些命令看似都够用了,但是它们的输出的效果有时并不是特别好看。而且有的时候,\LaTeX 提供的命令或环境并不能满足我们
的要求。试想:我要如何制作一个简单但像样的毕业论文/书籍/简历模板,每次都可以直接套用,而不是再在导言区写一堆代码?

这一章的内容可以帮我们实现这一目标,让我们编写可重复利用的模块——宏包和文档类,并在其中自己定义命令和环境,让 \LaTeX 产
生不同于默认的输出。

\section{自定义命令、环境和宏包}
\subsection{定义新命令}
定义新命令:
\begin{minted}{LaTeX}
    \newcommand{name}[num]{definition}
\end{minted}

该命令需要两个基本的参数:新命令的名字 \emph{name} 和 新命令的定义 \emph{definition}。可选参数 \emph{num} 可以
指定新命令的参数数目(最多9个)。如果不写 \emph{num} 参数,默认为0,表示新命令没有参数。

示例 1 定义了一个新命令 \mintinline{LaTeX}{\tnss},是“The Not So Short Introduction to \LaTeXe” 的缩写。

示例代码1:
\begin{minted}{LaTeX}
    \newcommand{\tnss}{The not so Short Introduction to \LaTeXe}
    % in the document body:
    This is "\tnss" \ldots{} "\tnss"
\end{minted}

示例输出1:
\begin{quote}
    \newcommand{\tnss}{The not so Short Introduction to \LaTeXe}
    % in the document body:
    This is "\tnss" \ldots{} "\tnss"
\end{quote}

示例 2 定义了包含参数的命令。使用定义的命令的时候,在 \texttt{\#1} 的位置指定参数。如果需要定义更多的参数,使用
\texttt{\#2}、\texttt{\#3},以此类推。

示例代码2:
\begin{minted}{LaTeX}
    \newcommand{\txsit}[2]{This is the \emph{#1} #2 Introduction to \LaTeXe}
    % in the document body:
    \begin{itemize}
        \item \txsit{not so}{short}
        \item \txsit{very}{long}
    \end{itemize}
\end{minted}

示例输出2:
\begin{quote}
    \newcommand{\txsit}[2]{This is the \emph{#1} #2 Introduction to \LaTeXe}
    % in the document body:
    \begin{itemize}
        \item \txsit{not so}{short}
        \item \txsit{very}{long}
    \end{itemize}
\end{quote}

\LaTeX 不允许定义一个与现有命令重名的命令。如果要修改命令定义,使用 \mintinline{LaTeX}{\renewcommand} 命令。它
的用法与 \mintinline{LaTeX}{\newcommand} 相同。

在某些情况下,可能会用到 \mintinline{LaTeX}{\providecommand} 命令。在命令不存在时,它相当于
\mintinline{LaTeX}{\newcommand};在命令已经存在时,仍沿用原来已存在的定义。

\subsection{定义新环境}
和 \mintinline{LaTeX}{\newcommand} 命令类似,可以通过 \mintinline{LaTeX}{\newenvironment} 命令来定义环境:
\begin{minted}{LaTeX}
    \newenvironment{name}[num]{before}{after}
\end{minted}

同样地,\mintinline{LaTeX}{\newenvironment} 有一个可选的参数 \emph{num}。\emph{before} 参数中的内容将在此环
境包含的文本之前处理,\emph{after} 参数中的内容将在遇到 \mintinline{LaTeX}|\end{name}| 命令时处理。

示例代码:
\begin{minted}{LaTeX}
    \newenvironment{king}
        {\rule{1ex}{1ex} \hspace{\stretch{1}}}
        {\hspace{\stretch{1}} \rule{1ex}{1ex}}
    % in the document body:
    \begin{king}
        My humble subjects \ldots
    \end{king}
\end{minted}

示例输出:
\begin{quote}
    \newenvironment{king}
        {\rule{1ex}{1ex} \hspace{\stretch{1}}}
        {\hspace{\stretch{1}} \rule{1ex}{1ex}}
    % in the document body:
    \begin{king}
        My humble subjects \ldots
    \end{king}
\end{quote}

\emph{num} 参数的用法和 \mintinline{LaTeX}{\newcommand} 命令相同。同样,也不能定义一个现有的环境,如果想修改某
个现有的环境,使用 \mintinline{LaTeX}{\renewenvironment} 命令,语法和 \mintinline{LaTeX}{\newenvironment}
相同。

上面示例中的 \mintinline{LaTeX}{\rule}、\mintinline{LaTeX}{\stretch} 和 \mintinline{LaTeX}{\hspace} 命
令后面会介绍。



\end{document}
