%-*- coding: UTF-8 -*-
% Customizing.tex
%
\documentclass[UTF8]{ctexart}
\usepackage{geometry}
\geometry{a4paper, centering, scale=0.8}
\usepackage{minted}
\usepackage{hyperref}

\title{\heiti 第6章 \quad \LaTeX 的私人订制}
\author{\kaishu Du Ang \\ \texttt{du2ang233@gmail.com} }
\date{\today}


\begin{document}
\maketitle

\tableofcontents

\newpage

之前学到的一些命令看似都够用了,但是它们的输出的效果有时并不是特别好看。而且有的时候,\LaTeX 提供的命令或环境并不能满足我们
的要求。试想:我要如何制作一个简单但像样的毕业论文/书籍/简历模板,每次都可以直接套用,而不是再在导言区写一堆代码?

这一章的内容可以帮我们实现这一目标,让我们编写可重复利用的模块——宏包和文档类,并在其中自己定义命令和环境,让 \LaTeX 产
生不同于默认的输出。

\section{自定义命令、环境、宏包和文档类}
\subsection{定义新命令}
定义新命令:
\begin{minted}{LaTeX}
    \newcommand{name}[num]{definition}
\end{minted}

该命令需要两个基本的参数:新命令的名字 \emph{name} 和 新命令的定义 \emph{definition}。可选参数 \emph{num} 可以
指定新命令的参数数目(最多9个)。如果不写 \emph{num} 参数,默认为0,表示新命令没有参数。

示例 1 定义了一个新命令 \mintinline{LaTeX}{\tnss},是“The Not So Short Introduction to \LaTeXe” 的缩写。

示例代码1:
\begin{minted}{LaTeX}
    \newcommand{\tnss}{The not so Short Introduction to \LaTeXe}
    % in the document body:
    This is "\tnss" \ldots{} "\tnss"
\end{minted}

示例输出1:
\begin{quote}
    \newcommand{\tnss}{The not so Short Introduction to \LaTeXe}
    % in the document body:
    This is "\tnss" \ldots{} "\tnss"
\end{quote}

示例 2 定义了包含参数的命令。使用定义的命令的时候,在 \texttt{\#1} 的位置指定参数。如果需要定义更多的参数,使用
\texttt{\#2}、\texttt{\#3},以此类推。

示例代码2:
\begin{minted}{LaTeX}
    \newcommand{\txsit}[2]{This is the \emph{#1} #2 Introduction to \LaTeXe}
    % in the document body:
    \begin{itemize}
        \item \txsit{not so}{short}
        \item \txsit{very}{long}
    \end{itemize}
\end{minted}

示例输出2:
\begin{quote}
    \newcommand{\txsit}[2]{This is the \emph{#1} #2 Introduction to \LaTeXe}
    % in the document body:
    \begin{itemize}
        \item \txsit{not so}{short}
        \item \txsit{very}{long}
    \end{itemize}
\end{quote}

\LaTeX 不允许定义一个与现有命令重名的命令。如果要修改命令定义,使用 \mintinline{LaTeX}{\renewcommand} 命令。它
的用法与 \mintinline{LaTeX}{\newcommand} 相同。

在某些情况下,可能会用到 \mintinline{LaTeX}{\providecommand} 命令。在命令不存在时,它相当于
\mintinline{LaTeX}{\newcommand};在命令已经存在时,仍沿用原来已存在的定义。

\subsection{定义新环境}
和 \mintinline{LaTeX}{\newcommand} 命令类似,可以通过 \mintinline{LaTeX}{\newenvironment} 命令来定义环境:
\begin{minted}{LaTeX}
    \newenvironment{name}[num]{before}{after}
\end{minted}

同样地,\mintinline{LaTeX}{\newenvironment} 有一个可选的参数 \emph{num}。\emph{before} 参数中的内容将在此环
境包含的文本之前处理,\emph{after} 参数中的内容将在遇到 \mintinline{LaTeX}|\end{name}| 命令时处理。

示例代码:
\begin{minted}{LaTeX}
    \newenvironment{king}
        {\rule{1ex}{1ex} \hspace{\stretch{1}}}
        {\hspace{\stretch{1}} \rule{1ex}{1ex}}
    % in the document body:
    \begin{king}
        My humble subjects \ldots
    \end{king}
\end{minted}

示例输出:
\begin{quote}
    \newenvironment{king}
        {\rule{1ex}{1ex} \hspace{\stretch{1}}}
        {\hspace{\stretch{1}} \rule{1ex}{1ex}}
    % in the document body:
    \begin{king}
        My humble subjects \ldots
    \end{king}
\end{quote}

\emph{num} 参数的用法和 \mintinline{LaTeX}{\newcommand} 命令相同。同样,也不能定义一个现有的环境,如果想修改某
个现有的环境,使用 \mintinline{LaTeX}{\renewenvironment} 命令,语法和 \mintinline{LaTeX}{\newenvironment}
相同。

上面示例中的 \mintinline{LaTeX}{\rule}、\mintinline{LaTeX}{\stretch} 和 \mintinline{LaTeX}{\hspace} 命
令后面会介绍。

\subsection{多余的空格}
在定义一个新环境的时候,很容易会有多余的空格“悄悄溜进来”。这样的小瑕疵有些时候却很致命,例如我们想创建一个没有缩进的标
题环境、并且希望紧跟这个环境后面的段落也不要缩进的时候。\mintinline{LaTeX}{\ignorespaces} 命令会忽略环境开始部分
的所有空格。想在环境结束后不留空格有点麻烦,可以用 \mintinline{LaTeX}{\ignorespacesafterend} 命令,它的作用是在
环境的‘end’之后执行 \mintinline{LaTeX}{\ignorespaces}。
示例代码:
\begin{minted}{LaTeX}
    \newenvironment{simple}
        {\noindent}
        {\par\noindent} % \par ends the paragraph
    % in the document body:
    \begin{simple}
        See the space \\ to the left.
    \end{simple}
    Same \\ here.

    \newenvironment{correct}
        {\noindent\ignorespaces}
        {\par\noindent\ignorespacesafterend}
    % in the document body:
    \begin{correct}
        No space \\ to the left
    \end{correct}
    Same \\ here.
\end{minted}

示例输出:
\begin{quote}
    \newenvironment{simple}
        {\noindent}
        {\par\noindent} % \par ends the paragraph
    % in the document body:
    \begin{simple}
        See the space \\ to the left.
    \end{simple}
    Same \\ here.

    \newenvironment{correct}
        {\noindent\ignorespaces}
        {\par\noindent\ignorespacesafterend}
    % in the document body:
    \begin{correct}
        No space \\ to the left
    \end{correct}
    Same \\ here.
\end{quote}

\subsection{\LaTeX 的命令行参数}
如果用的是类 Unix 操作系统,就可能会用 Makefile 来构建 \LaTeX 项目,然后就可以用不同的命令行参数使使同一份文档编译
出不同的版本。如果在文档中加入下面的代码:
\begin{minted}{LaTeX}
    \usepackage{ifthen}
    \ifthenelse{\equal{\blackandwhite}{true}}{
        % "black and white" mode; do something...
    }{
        % "color" mode; do something different...
    }
\end{minted}

然后利用下面的命令调用 \LaTeX:
\begin{minted}{LaTeX}
    latex '\newcommand{blackandwhite}{true}\input{test.tex}'
\end{minted}

首先 \mintinline{LaTeX}{\blackandwhite} 命令会被定义,然后真实的文件会被读取。如果将
\mintinline{LaTeX}{\blackandwhite}设置为 false,文档就会输出彩色版本。

\subsection{定义新宏包}
如果定义了很多的新环境和新命令,文档的导言区就会很长。这时,就可以定义一个 \LaTeX 宏包来包含所有新环境和新命令的定义。
需要时,再在文档中通过 \mintinline{LaTeX}{\usepackage} 命令调用。

如果想要自定义一个宏包,基本的工作是要将原来写在文档导言区的很长的内容拷贝到一个 \texttt{.sty} 文件中,并且需要在文件
最前面加上 \mintinline{LaTeX}|\ProviedesPackage{package name}| 命令,这个命令可以在多次包含宏包的问题提示错
误。其中,\emph{package name} 要和我们定义的宏包名相同。

示例代码:
\begin{minted}{LaTeX}
    % Demo Package by Tobias Oetiker
    \ProvidesPackage{demopack}
    \newcommand{\tnss}{The not so Short Introduction to \LaTeXe}
    \newcommand{\txsit}[1]{The \emph{#1} Short Introduction to \LaTeXe}
    \newenvironment{king}{\begin{quote}}{\end{quote}}
\end{minted}

如果想进一步把各种宏包的功能汇总到一个文件里,而不是在文档的导言区罗列一大堆宏包的话,\LaTeX 允许我们在自己编写的宏包
中调用其他宏包,命令为 \mintinline{LaTeX}{\RequirePackage},用法和 \mintinline{LaTeX}{\usepackage} 一致:
\begin{minted}{LaTeX}
    \RequirePackage[options]{package name}
\end{minted}

\subsection{定义新的文档类}
再进一步,如果想编写自己的文档类,如论文模板等,问题就稍稍麻烦一些了。首先,要把自定义的文档类文件以 \texttt{.cls} 作
为扩展名,开头使用 \mintinline{LaTeX}{\ProvidesClass} 命令:
\begin{minted}{LaTeX}
    \ProvidesClass{class name}
\end{minted}

同样地,\emph{class name} 也要和文档类的文件名一致。

但是有了上述命令和之前学到的一些命令,还不足以完成一个文档类的编写。因为诸如 \mintinline{LaTeX}{\chapter}、
\mintinline{LaTeX}{\section} 等等许多命令都是在文档类中定义的。事实上,许多时候我们只需要像调用宏包那样调用一个基
本的文档类,这样可以省去许多不必要的麻烦。在文档类中使用其他文档类的命令是 \mintinline{LaTeX}{\LoadClass},用法和
\mintinline{LaTeX}{\documentclass} 十分相像:
\begin{minted}{LaTeX}
    \LoadClass[options]{package name}
\end{minted}


\end{document}
